\documentclass[10pt,a4paper]{article}
\usepackage[right=0.5cm, left=0.5cm,top=1cm,bottom=1.5cm]{geometry}
\usepackage{enumitem}
\usepackage{graphicx}
\usepackage{array, tasks}
\usepackage{blindtext}
\usepackage{fontspec}
\usepackage{amsmath,amsfonts,amssymb,mathrsfs,amsthm}
\usepackage{fancyhdr}
\usepackage{xcolor}
\usepackage{booktabs}
\usepackage[font={bf}]{caption}
% \captionsetup[table]{box=colorbox,boxcolor=orange!20}
\usepackage{float}
\usepackage{esvect}
\usepackage{tabularx}
\usepackage{pifont}
\usepackage{colortbl}
 \usepackage{fancybox}
 \mathversion{bold}
 \usepackage{pgfplots}
 % \usepackage[utf8]{inputenc}
\usepackage{tikz}
 \usepackage[tikz]{bclogo}%
 \usepackage{mathpazo}
\usepackage{ulem}
\usepackage{yagusylo}
\usepackage{textcomp}\usepackage{blindtext}
\usepackage{multicol}
\usepackage{varwidth}
\usetikzlibrary{calc,intersections}
\usepackage{pgfplots}
%\usepackage{fourier}
\pgfplotsset{compat=1.11}
\usepackage{tkz-tab}
\usepackage{xcolor}
\usepackage{color}
\usetikzlibrary{calc}
\mathchardef\times="2202
\usepackage[most]{tcolorbox}
\definecolor{lightgray}{gray}{0.9}
\definecolor{ocre}{RGB}{0,244,244} 
\definecolor{head}{RGB}{255,211,204}
\definecolor{browndark}{RGB}{105,79,56}
%\RequirePackage[framemethod=default]{mdframed}
\usepackage{tikz}
\usetikzlibrary{calc,patterns,decorations.pathmorphing,arrows.meta,decorations.markings}
\usetikzlibrary{arrows.meta}
\makeatletter
\tcbuselibrary{skins,breakable,xparse}
\tcbset{%
  save height/.code={%
    \tcbset{breakable}%
    \providecommand{#1}{2cm}%
    \def\tcb@split@start{%
      \tcb@breakat@init%
      \tcb@comp@h@page%
      \def\tcb@ch{%
        \tcbset{height=\tcb@h@page}%
        \tcbdimto#1{#1+\tcb@h@page-\tcb@natheight}%
        \immediate\write\@auxout{\string\gdef\string#1{#1}}%
        \tcb@ch%
      }%
      \tcb@drawcolorbox@standalone%
    }%
  }%
}
\newcommand{\Lim}{\displaystyle\lim}
\makeatother
\newcommand{\oij}{$\left( \text{O};\vv{i},\vv{j} , \vv{k}\right)$}
\colorlet{darkred}{red!30!black}
\newcommand{\red}[1]{\textcolor{darkred}{ #1}}
\newcommand{\rr}{\mathbb{R}}
\renewcommand{\baselinestretch}{1.2}
 \setlength{\arrayrulewidth}{1.25pt}
\usepackage{titlesec}
\usepackage{titletoc}
\usepackage{minitoc}
\usepackage{ulem}
%--------------------------------------------------------------

\usetikzlibrary{decorations.pathmorphing}
\tcbuselibrary{skins}

%%%%%%%%%%%
%-------------------------------------------------------------------------
\tcbset{
        enhanced,
        colback=white,
        boxrule=0.1pt,
        colframe=brown!10,
        fonttitle=\bfseries
       }
\definecolor{problemblue}{RGB}{100,134,158}
\definecolor{idiomsgreen}{RGB}{0,162,0}
\definecolor{exercisebgblue}{RGB}{192,232,252}
\definecolor{darkbrown}{rgb}{0.4, 0.26, 0.13}

\newcommand*{\arraycolor}[1]{\protect\leavevmode\color{#1}}
\newcolumntype{A}{>{\columncolor{blue!50!white}}c}
\newcolumntype{B}{>{\columncolor{LightGoldenrod}}c}
\newcolumntype{C}{>{\columncolor{FireBrick!50}}c}
\newcolumntype{D}{>{\columncolor{Gray!42}}c}

\newcounter{mysection}
\newcounter{mysubsection}
\newcommand{\mysection}[1]{%
    \stepcounter{mysection} % Increment the counter
    \textcolor{red}{\LARGE\themysection. #1 :}
}
\newcommand{\mysubsection}[2]{
    \stepcounter{mysubsection}
    \textcolor{red}{\large \themysection.#1. #2 :}
}
% \textcolor{red}{\LARGE\bfseries 1. Les équation du deuxiéme degrée :}

%------------------------------------------------------
\newtcolorbox[auto counter]{Def}{enhanced,
before skip=2mm,after skip=2mm,
colback=yellow!20!white,colframe=lime,boxrule=0.2mm,
attach boxed title to top left =
    {xshift=0.6cm,yshift*=1mm-\tcboxedtitleheight},
    varwidth boxed title*=-3cm,
    boxed title style={frame code={
                        \path[fill=lime]
                            ([yshift=-1mm,xshift=-1mm]frame.north west)  
                            arc[start angle=0,end angle=180,radius=1mm]
                            ([yshift=-1mm,xshift=1mm]frame.north east)
                            arc[start angle=180,end angle=0,radius=1mm];
                        \path[left color=lime,right color = lime,
                            middle color = lime]
                            ([xshift=-2mm]frame.north west) -- ([xshift=2mm]frame.north east)
                            [rounded corners=1mm]-- ([xshift=1mm,yshift=-1mm]frame.north east) 
                            -- (frame.south east) -- (frame.south west)
                            -- ([xshift=-1mm,yshift=-1mm]frame.north west)
                            [sharp corners]-- cycle;
                            },interior engine=empty,
                    },
fonttitle=\bfseries\sffamily,
title={Définition ~\thetcbcounter}}
%------------------------------------------------------
\newtcolorbox[auto counter]{Prop}{enhanced,
before skip=2mm,after skip=2mm,
colback=yellow!20!white,colframe=blue,boxrule=0.2mm,
attach boxed title to top left =
    {xshift=0.6cm,yshift*=1mm-\tcboxedtitleheight},
    varwidth boxed title*=-3cm,
    boxed title style={frame code={
                        \path[fill=blue]
                            ([yshift=-1mm,xshift=-1mm]frame.north west)  
                            arc[start angle=0,end angle=180,radius=1mm]
                            ([yshift=-1mm,xshift=1mm]frame.north east)
                            arc[start angle=180,end angle=0,radius=1mm];
                        \path[left color=blue,right color = blue,
                            middle color = blue]
                            ([xshift=-2mm]frame.north west) -- ([xshift=2mm]frame.north east)
                            [rounded corners=1mm]-- ([xshift=1mm,yshift=-1mm]frame.north east) 
                            -- (frame.south east) -- (frame.south west)
                            -- ([xshift=-1mm,yshift=-1mm]frame.north west)
                            [sharp corners]-- cycle;
                            },interior engine=empty,
                    },
fonttitle=\bfseries\sffamily,
title={Proposition ~\thetcbcounter}}
%------------------------------------------------------
\newtcolorbox[auto counter]{Thm}{enhanced,
before skip=2mm,after skip=2mm,
colback=yellow!20!white,colframe=red,boxrule=0.2mm,
attach boxed title to top left =
    {xshift=0.6cm,yshift*=1mm-\tcboxedtitleheight},
    varwidth boxed title*=-3cm,
    boxed title style={frame code={
                        \path[fill=red]
                            ([yshift=-1mm,xshift=-1mm]frame.north west)  
                            arc[start angle=0,end angle=180,radius=1mm]
                            ([yshift=-1mm,xshift=1mm]frame.north east)
                            arc[start angle=180,end angle=0,radius=1mm];
                        \path[left color=red,right color = red,
                            middle color = red]
                            ([xshift=-2mm]frame.north west) -- ([xshift=2mm]frame.north east)
                            [rounded corners=1mm]-- ([xshift=1mm,yshift=-1mm]frame.north east) 
                            -- (frame.south east) -- (frame.south west)
                            -- ([xshift=-1mm,yshift=-1mm]frame.north west)
                            [sharp corners]-- cycle;
                            },interior engine=empty,
                    },
fonttitle=\bfseries\sffamily,
title={Théorème ~\thetcbcounter}}
%------------------------------------------------------
\newtcolorbox[auto counter]{exemple}{
  % breakable,
  enhanced,
  colback=white,
  boxrule=0pt,
  arc=0pt,
  outer arc=0pt,
  title=Exemple ~\thetcbcounter,
  fonttitle=\bfseries\sffamily\large\strut,
  coltitle=problemblue,
  colbacktitle=problemblue,
  title style={
left color=exercisebgblue,
    right color=white,
    middle color=exercisebgblue  
  },
  overlay={
    \draw[line width=1pt,problemblue] (frame.south west) -- (frame.south east);
    % \draw[line width=1pt,problemblue] (frame.north west) -- (frame.north east);
    \draw[line width=1pt,problemblue] (frame.south west) -- (frame.north west);
    % \draw[line width=1pt,problemblue] (frame.south east) -- (frame.north east);
  }
}
%----------------------------------------------------
\newtcolorbox[auto counter]{Activite}{
  % breakable,
  enhanced,
  colback=white,
  boxrule=0pt,
  arc=0pt,
  outer arc=0pt,
  title=Activité ~\thetcbcounter,
  fonttitle=\bfseries\sffamily\large\strut,
  coltitle=problemblue,
  colbacktitle=problemblue,
  title style={
left color=yellow!50!white,
    right color=white,
    middle color=yellow!20!white  
  },
  overlay={
    \draw[line width=1pt,problemblue] (frame.south west) -- (frame.south east);
    \draw[line width=1pt,problemblue] (frame.north west) -- (frame.north east);
    \draw[line width=1pt,problemblue] (frame.south west) -- (frame.north west);
    \draw[line width=1pt,problemblue] (frame.south east) -- (frame.north east);
  }
}
%----------------------------------------------------
\newtcolorbox[auto counter]{application}{
  % breakable,
  enhanced,
  colback=white,
  boxrule=0pt,
  arc=0pt,
  outer arc=0pt,
  title=Application ~\thetcbcounter,
  fonttitle=\bfseries\sffamily\large\strut,
  coltitle=problemblue,
  colbacktitle=problemblue,
  title style={
left color=exercisebgblue,
    right color=white,
    middle color=exercisebgblue  
  },
  overlay={
    \draw[line width=1pt,problemblue] (frame.south west) -- (frame.south east);
    \draw[line width=1pt,problemblue] (frame.north west) -- (frame.north east);
    \draw[line width=1pt,problemblue] (frame.south west) -- (frame.north west);
    \draw[line width=1pt,problemblue] (frame.south east) -- (frame.north east);
  }
}
%----------------------------------------------------
\newtcolorbox{mybox}[2]{enhanced,breakable,
    before skip=2mm,after skip=2mm,
    colback=white,colframe=#2!30!blue,boxrule=0.3mm,rightrule=0.3mm,
    attach boxed title to top center={xshift=0cm,yshift*=1mm-\tcboxedtitleheight},
    varwidth boxed title*=-3cm,
    boxed title style={frame code={
    \path[fill=#2!30!black]
    ([yshift=-1mm,xshift=-1mm]frame.north west)
    arc[start angle=0,end angle=180,radius=1mm]
    ([yshift=-1mm,xshift=1mm]frame.north east)
    arc[start angle=180,end angle=0,radius=1mm];
    \path[draw=black,line width=1pt,left color=#2!1!white,right color=#2!1!blue!65,
    middle color=#2!1!green]
    ([xshift=-2mm]frame.north west) -- ([xshift=2mm]frame.north east)
    [rounded corners=1mm]-- ([xshift=1mm,yshift=-1mm]frame.north east)
    -- (frame.south east) -- (frame.south west)
    -- ([xshift=-1mm,yshift=-1mm]frame.north west)
    [sharp corners]-- cycle;
    },interior engine=empty,
    },
title=#1,coltitle=black,fonttitle=\sffamily}
%---------------------------------------------
\newtcolorbox{boxone}{%
    enhanced,
    colback=brown!10,
    boxrule=0pt,
    sharp corners,
    drop lifted shadow,
    frame hidden,
    fontupper=\bfseries,
    notitle,
    overlay={%
        \draw[Circle-Circle, brown!70!black, line width=2pt](frame.north west)--(frame.south west); 
        \draw[Circle-Circle, brown!70!black, line width=2pt](frame.north east)--(frame.south east);}
    }
    
\begin{document}


\begin{tcolorbox}[title=\textcolor{blue}{\shadowbox{ Prof : Othmane Laksoumi}}
\hfill
\textcolor{blue}{\shadowbox{ Ensemble des nombres entiers naturels $\mathbb{N}$ et notions en arithmétique
 }}]

\end{tcolorbox}

\begin{mybox}{Lycée Qualifiant Zitoun}{gray}
    \begin{minipage}{8cm}
    \textcolor{darkbrown}{Année scolaire : } 2024-2025 \\
    \textcolor{darkbrown}{Niveau : } Tronc commun scientifique \\
    \textcolor{darkbrown}{Durée totale : } $7h$
    \end{minipage}
\end{mybox}

\begin{boxone}
{\Large\ding{45}}
\textcolor{red}{\large Contenus du programme :}
\begin{itemize}
    \item Les nombres pairs et les nombres impairs
    \item  Multiples d’un nombre, le plus petit multiple commun de deux nombres
    \item Diviseurs d’un nombre, le plus grand diviseur commun de deux nombres
    \item Nombres premiers, décomposition d’un nombre en produit de facteurs
premiers
\end{itemize}

{\Large\ding{45}}
\textcolor{red}{\large Les capacités attendues :}
\begin{itemize}
    \item Utiliser la parité et la décomposition en produit de facteurs premiers pour résoudre des problèmes simples portant les entiers naturels.
\end{itemize}

{\Large\ding{45}}
\textcolor{red}{\large Recommandations pédagogiques :} 
  \begin{itemize}
      \item  On introduira les symboles : $\in,\ \notin,\ \subset,\ \not\subset,\ \bigcup,\ \bigcap$
      \item  l’objectif de la présentation de "notions en
arithmétique" est d’initier les élèves à des modes de
démonstration à travers l’utilisation des nombres pairs
et des nombres impairs sans excès.
  \end{itemize}
\end{boxone}

\newpage

\begin{tabular}{|>{\centering\arraybackslash}p{1.2cm}|>{\raggedright\arraybackslash}p{15.5cm}|>{\centering\arraybackslash}p{0.8cm}|}
\hline
\rowcolor{head}

Etapes &
\centering Contenu du cours &
 Durée \\
\hline

% \vspace{1cm}
% \rotatebox{90}{Phase de lancement}

% \vspace{1cm}

% \rotatebox{90}{construction de connaissances}
 
&
\vspace{0.1cm}

\mysection{Nombres pairs et nombres impaires}
\begin{Activite}
\begin{enumerate}
    \item Parmi les nombres suivants, déterminer les entiers naturels :
        $$10 \hspace{10mm} \displaystyle\frac{3}{2} \hspace{10mm} -5 \hspace{10mm} \displaystyle\frac{10}{2} \hspace{10mm} \sqrt{2} \hspace{10mm} \sqrt{25} \hspace{10mm} -1$$
    \item Parmi les entiers naturels suivantes, déterminer les multiples du nombre 2 :
        $$4 \hspace{10mm} 19 \hspace{10mm} 15+25 \hspace{10mm} 2^3-1 \hspace{10mm} 44 \hspace{10mm} 3^3+1$$
\end{enumerate}
\end{Activite}



\begin{Def}
\begin{itemize}
    \item Tout entier naturel multiple de $2$ est appelé nombre \textcolor{red}{pair}.
    \item Tout entier naturel qui n'est pas pair est dit \textcolor{red}{impair}.
    \item Les nombres pairs sont les nombres qui s'écrivent sous la forme $2k$ où $k$ est un nombre entier naturel.
    \item Les nombres impairs sont les nombres qui s'écrivent sous la forme $2k+1$ où $k$ est un nombre entier naturel, ou sous la forme $2k-1$ où $k$ est un nombre entier naturel non nul.
\end{itemize}
\end{Def}

\begin{exemple}
    \begin{enumerate}
        \item $2004$ est un nombre pair.
        \item $2005$ est un nombre impair.
        \item Soit $x$ un entier naturel non nul et différent de $1$.
        $A = 2x-3$ et $B = 4x + 2$
    \end{enumerate}
\end{exemple}

\begin{application}
Soit $n$ un entier naturel. Etudier la parité de $A$ et $B$ tels que : $A = 2n^2 + 6$ et $B = 8n+3$
\end{application}
\textcolor{red}{Remarque :} \newline
Pour qu'un entier naturel soit pair, il suffit que son chiffre d'unités soit 0, 2, 4, 6 ou 8. \newline\newline
\mysection{Opérations sur les nombres pairs et impairs}
\begin{Prop}
    Soient $a$ et $b$ deux entiers naturels tels que $a\geq b$.
    \begin{itemize}
        \item Si $a$ et $b$ sont pairs, alors $a+b$ et $a-b$ sont pairs.
        \item Si $a$ et $b$ sont impairs, alors $a+b$ et $a-b$ sont pairs.
        \item Si l'un des deux nombres $a$ et $b$ pair et l'autre impair, alors $a+b$ et $a-b$ sont impairs.
        \item Si l'un des deux nombres $a$ et $b$ pair, alors $ab$ est pair (quelle que soit la parité de l'autre).
        \item Si $a$ et $b$ sont impairs, alors $ab$ est impair.
    \end{itemize}
\end{Prop}


    

&
 \\
\hline

\end{tabular}

\begin{tabular}{|>{\centering\arraybackslash}p{1.2cm}|>{\raggedright\arraybackslash}p{15.5cm}|>{\centering\arraybackslash}p{0.8cm}|}
\hline

&
\vspace{1mm}
\begin{application}
    Soit $n$ un entier naturel. Etudier la parité des entiers naturels suivants :\newline $A = 4n^2+19$, $B = 10n^3+5n^2+1$ et $C = n(n+1)$
\end{application}

\mysection{Multiples d'un nombre entier naturel}
\begin{Activite}
Cocher les réponses justes : \newline
    \begin{tabular}{|>{\centering}p{3cm}|p{0.5cm}|p{0.5cm}|p{0.5cm}|p{0.5cm}|p{0.5cm}|p{0.5cm}|p{0.5cm}|}
    \hline
         & 6 & 21 & 14 & 111 & 15 & 18 \\
    \hline
        Multiple de $3$ & & &  & & & \\
    \hline
        Multiple de $5$ & & &  & & & \\
    \hline
        Multiple de $7$ & & &  & & & \\
    \hline
    \end{tabular}
\end{Activite}
    
\begin{Def}
    Soient $m$ et $n$ deux entiers naturels. On dit que $m$ est un multiple de $n$ si $m=n\times k$ où $k$ un entier naturel.
\end{Def}

\begin{exemple}
    \begin{multicols}{2}
        \begin{itemize}
            \item $42$ est un multiple de $21$ car $42 = 2\times 21$
            \item $55$ est un multiple de $11$ car $55 = 5\times 11$
        \end{itemize}
    \end{multicols}
\end{exemple}

\textcolor{red}{Remarque :}
\begin{itemize}
    \item Tout entier naturel $a$ est un multiple de lui-même et de $a$.
    \item $0$ est un multiple de tous les entiers naturels.
    \item Les multiples d'un entier naturel $a$ sont : $0,a,2a,3a,\dots,100a, \dots$
\end{itemize}

\begin{application}
    \begin{enumerate}
        \item Montrer que $15\times 18$ est un multiple de $30$.
        \item Déterminez les multiples de $7$ inférieurs à $60$.
    \end{enumerate}
\end{application}

\mysection{Diviseurs d'un entier naturel}
\begin{Def}
    Soient $a$ et $b$ deux entiers naturels. On dit que $a$ est un diviseur de $b$ si $b$ est un multiple de $a$ c'est-à-dire $b=ak$ où $k$ un entier naturel.\\
    Si $a$ est un diviseur de $b$, on dit aussi :
    \begin{itemize}
        \item $a$ divise $b$.
        \item $b$ est divisible par $a$.
        \item $b$ est un multiple de $a$.
    \end{itemize}
\end{Def}

\begin{exemple}
    \begin{multicols}{2}
        \begin{itemize}
            \item $6$ est un diviseur de $24$ car $24 = 6\times 4$
            \item $7$ est un diviseur de $77$ car $77 = 7\times 11$
        \end{itemize}
    \end{multicols}
\end{exemple}

& \\
\hline

\end{tabular}

\begin{tabular}{|>{\centering\arraybackslash}p{1.2cm}|>{\raggedright\arraybackslash}p{15.5cm}|>{\centering\arraybackslash}p{0.8cm}|}
\hline

&
\vspace{1mm}
\textcolor{red}{Remarque :}
    \begin{itemize}
        \item $1$ est un diviseur de tout entier.
        \item Si un entier $a$ divise un entier $b$, alors $a\leq b$.
\end{itemize}


\begin{application}
    Déterminer tous les diviseurs de $30$.
\end{application}

\begin{Prop}
    Soient $a,b$ et $c$ des entiers naturels.
    \begin{itemize}
        \item Si $a$ divise $b$ et $c$, et $b\geq c$ alors $a$ divise $b+c$ et $b-c$.
        \item Si $a$ divise $b$, alors $a$ divise $bc$.
    \end{itemize}
\end{Prop}

\mysection{Nombres premier}
\begin{Activite}
    \begin{enumerate}
        \item Déterminez les diviseurs des entiers naturels suivants : 
        $$2 \hspace{10mm} 3 \hspace{10mm} 5 \hspace{10mm} 7 \hspace{10mm} 11 \hspace{10mm} 13 \hspace{10mm} 41$$
        \item Que remarquez-vous?
    \end{enumerate}
\end{Activite}

\begin{Def}
    Un entier naturel $p$ est dit premier s'il admet \textcolor{red}{exactement deux diviseurs} différents.
\end{Def}

\begin{exemple}
    \begin{enumerate}
        \item 31 est un nombre premier.
    \end{enumerate}
\end{exemple}

\textcolor{red}{Remarque :}
\begin{itemize}
    \item 1 n'est pas un nombre premier parce qu'il n'admet qu'un seul diviseur.
    \item 2 est le seul nombre premier pair.
    \item Tout nombre premier différent de 2 est impair.
\end{itemize}

\mysection{Décomposition d’un nombre non premier en produit de facteurs premiers}
\begin{Prop}
    Tout entier naturel non premier et supérieur à 1 peut être décomposé en produit de facteurs premiers.
\end{Prop}

\begin{exemple}
    \begin{enumerate}
        \item L'écriture $2^2\times 3\times 5$ est la décomposition du nombre $60$ en produit de facteurs premiers.
    \end{enumerate}
\end{exemple}

\begin{application}
    Décomposer les nombres suivants en produits de facteurs premiers : $100$, $63$, $32$.
\end{application}





& \\
\hline

\end{tabular}


\begin{tabular}{|>{\centering\arraybackslash}p{1.2cm}|>{\raggedright\arraybackslash}p{15.5cm}|>{\centering\arraybackslash}p{0.8cm}|}
\hline

&
\vspace{1mm}
\mysection{Diviseurs communs de deux entiers naturels}
\begin{Def}
    On dit qu'un entier naturel $d$ est un diviseur commun des deux entiers naturels $a$ et $b$ si $d$ est un diviseur de $a$ et $b$.
\end{Def}
\begin{exemple}
    Les diviseurs de $12$ sont $1$, $2$, $3$, $4$, $6$, $12$. \\
    Les diviseurs de $18$ sont $1$, $2$, $3$, $6$, $9$, $18$. \\
    Les nombres $1$, $2$, $3$, $6$ sont les diviseurs communs de $12$ et $18$.
\end{exemple}

\begin{application}
    Déterminer les diviseurs communs de $18$ et $150$.
\end{application}

\mysection{Plus grand diviseur commun de deux entiers naturels}
\begin{Def}
    Le plus grand diviseur commun de deux entiers naturels $a$ et $b$ est le plus grand entier parmi les diviseurs communs de $a$ et $b$. On le note par $a\wedge b$ ou $a\Delta b$.
\end{Def}

\textcolor{red}{Remarque :}\newline
Pour dire que $a\wedge b$ est le plus grand commun diviseur de $a$ et $b$, on dit que $a\wedge b$ est le pgcd de $a$ et $b$.

\begin{exemple}
    Les diviseurs de $30$ sont $1$, $2$, $3$, $5$, $6$, $10$, $15$, $30$. \\
    Les diviseurs de $18$ sont $1$, $2$, $3$, $4$, $6$, $12$. \\
    Alors $a\wedge b = 6.$ 
\end{exemple}

\begin{application}
    Déterminer $15\wedge 75$, $13\wedge 14$, $27\wedge 36$.
\end{application}

\mysection{Multiples communs de deux entiers naturels}
\begin{Def}
    On dit qu'un entier naturel $m$ est un multiple commun des deux entiers naturels $a$ et $b$ si $m$ est un multiple de $a$ et $b$.
\end{Def}

\begin{exemple}
    $36$ est un multiple commun de $6$ et $4$ (car $36=9\times 4$ et $36 = 6\times 6)$.
\end{exemple}

\mysection{Plus petit multiple commun de deux entiers naturels}
\begin{Def}
    Le plus petit multiple commun de deux entiers naturels $a$ et $b$ est le plus petit multiple commun non nul de $a$ et $b$. On le note par : $a\vee b$ ou $M(a,b)$. 
\end{Def}

\textcolor{red}{Remarque :}
Pour dire que $a\vee b$ est le plus petit multiple commun de $a$ et $b$, on dit que $a\vee b$ est le ppcm de $a$ et $b$.



& \\
\hline

\end{tabular}

\begin{tabular}{|>{\centering\arraybackslash}p{1.2cm}|>{\raggedright\arraybackslash}p{15.5cm}|>{\centering\arraybackslash}p{0.8cm}|}
\hline

&
\vspace{1mm}

\begin{exemple}
    Les multiples non nuls de $4$ sont $4$, $8$, $12$, $16$, $20$, ...\\
    Les multiples non nuls de $6$ sont $6$, $12$, $18$, $24$, $30$, ...\\
    Alors $4\vee 6 = 12$.
\end{exemple}

\begin{application}
    Déterminer $15\vee 25$, $24\vee 18$, $5\vee 7$.
\end{application}

\begin{Activite}
    \begin{enumerate}
        \item Déterminer les diviseurs de $100$ et $120$ puis déduire $100\wedge 120$.
        \item Décomposer $100$ et $120$ en produits de facteurs premiers.
        \item Calculez le produit des facteurs premiers communs de $100$ et $120$ avec la plus petite puissance.
        \item Que remarquez-vous?
    \end{enumerate}
\end{Activite}

\begin{Thm}
\begin{enumerate}
    \item Le pgcd de deux entiers naturels est le produit des facteurs premiers communs de leurs décompositions affectés de leur plus petit exposant.
    \item Le ppcm de deux entiers naturels est le produit des facteurs premiers communs et non communs de leurs décompositions affectés de leur plus grand exposant.
\end{enumerate}
\end{Thm}

\begin{exemple}
    $a = 2^4\times 3\times 5\times 11$ et $b = 2^2\times 5^2\times 7$. $a\wedge b = 2^2\times 5$ et $a\vee b = 2^4\times 3\times 5^2\times 7\times 11$.
\end{exemple}

\begin{application}
    Déterminez $a\wedge b$ et $a\vee b$ dans les cas suivants :
    \begin{enumerate}
        \item $a= 14$ et $b = 26$.
        \item $a= 252$ et $b = 313$.
        \item $a= 14$ et $b = 26$.
    \end{enumerate}
\end{application}


&\\
\hline

\end{tabular}



\end{document}