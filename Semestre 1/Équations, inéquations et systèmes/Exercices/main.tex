\documentclass[12pt,a4paper]{article}
\usepackage[top=2cm,left=1cm,right=1cm,bottom=2cm]{geometry}
\usepackage{amssymb,mathtools,amsthm}
\usepackage{fourier}
\usepackage{xcolor}
\usepackage{multicol, array, fancyhdr}
\usepackage{tasks}
\newcommand{\Lim}{\displaystyle\lim}
\renewcommand{\columnseprule}{1pt}
\renewcommand{\arraystretch}{1.5}
% \renewcommand{\dfrac}[2]{\displaystyle\dfrac{#1}{#2}}


%======================================================
\newtheoremstyle{mystyle}
  {\topsep}% espace avant
  {\topsep}% espace après
  {\upshape}% police du corps du théorème
  {}% indentation (vide pour rien, \parindent)
  {\bfseries\sffamily}% police du titre du théorème
  { :}% ponctuation après le théorème
  { }% après le titre du théorème (espace ou \newline)
  {%
    \rule[0.5\baselineskip]{0.5\textwidth}{1pt}%
    \newline\fcolorbox{black}{white}{%
      \thmname{#1}\thmnumber{ \textup{#2}}\thmnote{ \textnormal{(#3)}}%
    }%
    
    % \vspace{0.5em} % Adjust vertical space after the title
  }% spécifications du titre

\theoremstyle{mystyle}
\newtheorem{exo}{Exercice}

%======================================================



\begin{document}


\pagestyle{fancy}
\fancyhf{} % clear all header and footer fields
\fancyhead[L]{Lycée : Zitoun \hspace{1.5cm} Année scolaire : 2024-2025} % Left header
\fancyhead[C]{ \hspace{4cm} Niveau : TCS} % Right-Center header
\fancyhead[R]{Prof. Othmane Laksoumi} % Right header
\fancyfoot[C]{\thepage} % Footer


\begin{center}
    \textbf{\Large Équations, inéquations et systèmes}
\end{center}
\begin{multicols*}{2}

\begin{exo}
\text{ }
\begin{enumerate}
	\item Dresser le tableau de signe de : 
	
	$-4x + 8$, $2x - 3$, $(5x - 2)(3x + 1)$ et $\dfrac{-4x + 8}{6x + 2}$
	\item Résoudre dans $\mathbb{R}^2$ les systèmes suivants :
	$$\begin{cases} 
5x - 2y = 4 \\ 
-3x + 4y = 6 
\end{cases}
\begin{cases} 
10x + y = 7 \\ 
100x + y = 1 
\end{cases}$$
$$\begin{cases} 
3x + 4y = 1 \\ 
x - 2y = 2 
\end{cases}
\begin{cases} 
3x - 2y = 6 \\ 
\dfrac{x}{2} - \dfrac{y}{3} = 0 
\end{cases}
$$
\end{enumerate}
\end{exo}

\begin{exo}
Résoudre dans \(\mathbb{R}\) les équations suivantes :
%\begin{multicols}{2}
\begin{enumerate}
    \item \(2x^{2} + x - 1 = 0\)
    \item \(x^{2} - 8x + 16 = 0\)
    \item \(-2x^{2} + 2\sqrt{2}x - 1 = 0\)
    \item \(x^{2} - 2x - 3 = 0\)
    \item \(x^{4} - 4\sqrt{2}x^{2} + 6 = 0\)
    \item \(3x^{4} - 4\sqrt{3}x^{2} + 4 = 0\)
    \item \(-3x^2 + 5x - 4 = 0\)
\end{enumerate}

%\end{multicols}
\end{exo}

\begin{exo}
Factoriser si c'est possible les trinômes suivants :
\begin{enumerate}
	\item \(P_{1}(x) = 2x^{2} + 3x - 2\)
	\item \(P_{2}(x) = -3x^{2} + 7x - 2\)
	\item \(P_{3}(x) = 25x^{2} - 10x + 1\)
	\item \(P_{4}(x) = 3x^{2} + 6\sqrt{3}x + 9\)
	\item \(P_{5}(x) = -3x^{4} + x^{2} - 2\)
	\item \(P_{5}(x) = -3x^{2} + x - 2\)
\end{enumerate}
\end{exo}

\newcolumn

\begin{exo}
Étudier le signe du trinôme \(P(x)\) :
\begin{enumerate}
    \item \(P(x) = x^{2} - 7x + 12\)
    \item \(P(x) = -x^{2} + 6x - 9\)
    \item \(P(x) = -x^{2} + 2x - 3\)
    \item \(P(x) = 7x^{2} + 12x + 5\)
    \item \(P(x) = x^{2} + x + 1\)
\end{enumerate}
\end{exo}

\begin{exo}
Résoudre dans \(\mathbb{R}\) les inéquations :
\begin{enumerate}
	\item \(x^{2} - 5x + 6 \geq 0\)
	\item \(x^{2} - 8x + 5 < 0\)
	\item \(49x^{2} - 70x + 25 > 0\)
	\item \(-x^{2} + x + 3 < 0\)
	\item \(-9x^{2} + 6\sqrt{2}x - 2 > 0\)
	\item \(\dfrac{x^{2} - 6x + 9}{3x^{2} + 10x - 8} \geq 0\)
%	\item \(4x - 3\sqrt{x} + 1 < 0\)
	\item \(\dfrac{x^{2} - 8x + 9}{x^{2} - 4x} < 0\)
	\item \(\dfrac{x^{2} - 8}{x^{2} + 12x - 13} \geq 0\)
%	\item \(\dfrac{2}{x + 1} + \dfrac{1}{x^{2} + x} \leq \dfrac{3(x + 3)}{x}\)
	\item \(-\dfrac{2x^{2} + x + 1}{x^{2} - 4x - 5} \geq 0\)
\end{enumerate}
\end{exo}

\begin{exo}
On considère l'équation \[(E): -2x^{2} + \sqrt{2}x + 2 = 0\]

\begin{enumerate}
    \item Montrer que l'équation \((E)\) admet deux \\ solutions distinctes \(\alpha\) et \(\beta\) (sans les calculer).
    \item Calculer \(\alpha + \beta\), \(\alpha \times \beta\), \(\dfrac{1}{\alpha} + \dfrac{1}{\beta}\), \(\alpha^{2}\beta + \alpha\beta^{2}\), \(\alpha^{2} + \beta^{2}\), \(\dfrac{\alpha}{\beta} + \dfrac{\beta}{\alpha}\) et \(\alpha^{3} + \beta^{3}\).
\end{enumerate}
\end{exo}




























\end{multicols*}



\end{document}
