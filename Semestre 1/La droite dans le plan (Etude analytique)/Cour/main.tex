\documentclass[10pt,a4paper]{article}
\usepackage[right=0.5cm, left=0.5cm,top=1cm,bottom=1.5cm]{geometry}
\usepackage{enumitem}
\usepackage{graphicx}
\usepackage{array, tasks}
\usepackage{blindtext}
\usepackage{fontspec}
\usepackage{amsmath,amsfonts,amssymb,mathrsfs,amsthm}
\usepackage{fancyhdr}
\usepackage{xcolor}
\usepackage{booktabs}
\usepackage[font={bf}]{caption}
% \captionsetup[table]{box=colorbox,boxcolor=orange!20}
\usepackage{float}
\usepackage{esvect}
\usepackage{tabularx}
\usepackage{pifont}
\usepackage{colortbl}
 \usepackage{fancybox}
 \mathversion{bold}
 \usepackage{pgfplots}
 % \usepackage[utf8]{inputenc}
\usepackage{tikz}
 \usepackage[tikz]{bclogo}%
 \usepackage{mathpazo}
\usepackage{ulem}
\usepackage{yagusylo}
\usepackage{textcomp}
\usepackage{blindtext}
\usepackage{multicol}
\usepackage{varwidth}
\usetikzlibrary{calc,intersections}
\usepackage{pgfplots}
%\usepackage{fourier}
\pgfplotsset{compat=1.11}
\usepackage{tkz-tab}
\usepackage{xcolor}
\usepackage{color}
\usetikzlibrary{calc}
\mathchardef\times="2202
\usepackage[most]{tcolorbox}
\definecolor{lightgray}{gray}{0.9}
\definecolor{ocre}{RGB}{0,244,244} 
\definecolor{head}{RGB}{255,211,204}
\definecolor{browndark}{RGB}{105,79,56}
%\RequirePackage[framemethod=default]{mdframed}
\usepackage{tikz}
\usetikzlibrary{calc,patterns,decorations.pathmorphing,arrows.meta,decorations.markings}
\usetikzlibrary{arrows.meta}
\makeatletter
\tcbuselibrary{skins,breakable,xparse}
\tcbset{%
  save height/.code={%
    \tcbset{breakable}%
    \providecommand{#1}{2cm}%
    \def\tcb@split@start{%
      \tcb@breakat@init%
      \tcb@comp@h@page%
      \def\tcb@ch{%
        \tcbset{height=\tcb@h@page}%
        \tcbdimto#1{#1+\tcb@h@page-\tcb@natheight}%
        \immediate\write\@auxout{\string\gdef\string#1{#1}}%
        \tcb@ch%
      }%
      \tcb@drawcolorbox@standalone%
    }%
  }%
}
\newcommand{\Lim}{\displaystyle\lim}
\makeatother
\newcommand{\oij}{$\left( \text{O};\vv{i},\vv{j} , \vv{k}\right)$}
\colorlet{darkred}{red!30!black}
\newcommand{\red}[1]{\textcolor{darkred}{ #1}}
\newcommand{\rr}{\mathbb{R}}
\renewcommand{\baselinestretch}{1.2}
 \setlength{\arrayrulewidth}{1.25pt}
\usepackage{titlesec}
\usepackage{titletoc}
\usepackage{minitoc}
\usepackage{ulem}
%--------------------------------------------------------------

\usetikzlibrary{decorations.pathmorphing}
\tcbuselibrary{skins}

%%%%%%%%%%%
%-------------------------------------------------------------------------
\tcbset{
        enhanced,
        colback=white,
        boxrule=0.1pt,
        colframe=brown!10,
        fonttitle=\bfseries
       }
\definecolor{problemblue}{RGB}{100,134,158}
\definecolor{idiomsgreen}{RGB}{0,162,0}
\definecolor{exercisebgblue}{RGB}{192,232,252}
\definecolor{darkbrown}{rgb}{0.4, 0.26, 0.13}

\newcommand*{\arraycolor}[1]{\protect\leavevmode\color{#1}}
\newcolumntype{A}{>{\columncolor{blue!50!white}}c}
\newcolumntype{B}{>{\columncolor{LightGoldenrod}}c}
\newcolumntype{C}{>{\columncolor{FireBrick!50}}c}
\newcolumntype{D}{>{\columncolor{Gray!42}}c}

\newcounter{mysection}
\newcounter{mysubsection}
\newcommand{\mysection}[1]{%
    \stepcounter{mysection} % Increment the counter
    \textcolor{red}{\LARGE\themysection. #1 :}
}
\newcommand{\mysubsection}[2]{
    \stepcounter{mysubsection}
    \textcolor{red}{\large \themysection.#1. #2 :}
}
% \textcolor{red}{\LARGE\bfseries 1. Les équation du deuxiéme degrée :}

%------------------------------------------------------
\newtcolorbox[auto counter]{Definition}{enhanced,
before skip=2mm,after skip=2mm,
colback=yellow!20!white,colframe=lime,boxrule=0.2mm,
attach boxed title to top left =
    {xshift=0.6cm,yshift*=1mm-\tcboxedtitleheight},
    varwidth boxed title*=-3cm,
    boxed title style={frame code={
                        \path[fill=lime]
                            ([yshift=-1mm,xshift=-1mm]frame.north west)  
                            arc[start angle=0,end angle=180,radius=1mm]
                            ([yshift=-1mm,xshift=1mm]frame.north east)
                            arc[start angle=180,end angle=0,radius=1mm];
                        \path[left color=lime,right color = lime,
                            middle color = lime]
                            ([xshift=-2mm]frame.north west) -- ([xshift=2mm]frame.north east)
                            [rounded corners=1mm]-- ([xshift=1mm,yshift=-1mm]frame.north east) 
                            -- (frame.south east) -- (frame.south west)
                            -- ([xshift=-1mm,yshift=-1mm]frame.north west)
                            [sharp corners]-- cycle;
                            },interior engine=empty,
                    },
fonttitle=\bfseries\sffamily,
title={Définition ~\thetcbcounter}}
%------------------------------------------------------
\newtcolorbox[auto counter]{Proposition}{enhanced,
before skip=2mm,after skip=2mm,
colback=yellow!20!white,colframe=blue,boxrule=0.2mm,
attach boxed title to top left =
    {xshift=0.6cm,yshift*=1mm-\tcboxedtitleheight},
    varwidth boxed title*=-3cm,
    boxed title style={frame code={
                        \path[fill=blue]
                            ([yshift=-1mm,xshift=-1mm]frame.north west)  
                            arc[start angle=0,end angle=180,radius=1mm]
                            ([yshift=-1mm,xshift=1mm]frame.north east)
                            arc[start angle=180,end angle=0,radius=1mm];
                        \path[left color=blue,right color = blue,
                            middle color = blue]
                            ([xshift=-2mm]frame.north west) -- ([xshift=2mm]frame.north east)
                            [rounded corners=1mm]-- ([xshift=1mm,yshift=-1mm]frame.north east) 
                            -- (frame.south east) -- (frame.south west)
                            -- ([xshift=-1mm,yshift=-1mm]frame.north west)
                            [sharp corners]-- cycle;
                            },interior engine=empty,
                    },
fonttitle=\bfseries\sffamily,
title={Proposition ~\thetcbcounter}}
%------------------------------------------------------
\newtcolorbox[auto counter]{Theoreme}{enhanced,
before skip=2mm,after skip=2mm,
colback=yellow!20!white,colframe=red,boxrule=0.2mm,
attach boxed title to top left =
    {xshift=0.6cm,yshift*=1mm-\tcboxedtitleheight},
    varwidth boxed title*=-3cm,
    boxed title style={frame code={
                        \path[fill=red]
                            ([yshift=-1mm,xshift=-1mm]frame.north west)  
                            arc[start angle=0,end angle=180,radius=1mm]
                            ([yshift=-1mm,xshift=1mm]frame.north east)
                            arc[start angle=180,end angle=0,radius=1mm];
                        \path[left color=red,right color = red,
                            middle color = red]
                            ([xshift=-2mm]frame.north west) -- ([xshift=2mm]frame.north east)
                            [rounded corners=1mm]-- ([xshift=1mm,yshift=-1mm]frame.north east) 
                            -- (frame.south east) -- (frame.south west)
                            -- ([xshift=-1mm,yshift=-1mm]frame.north west)
                            [sharp corners]-- cycle;
                            },interior engine=empty,
                    },
fonttitle=\bfseries\sffamily,
title={Théorème ~\thetcbcounter}}
%------------------------------------------------------
\newtcolorbox[auto counter]{Exemple}{
  % breakable,
  enhanced,
  colback=white,
  boxrule=0pt,
  arc=0pt,
  outer arc=0pt,
  title=Exemple ~\thetcbcounter,
  fonttitle=\bfseries\sffamily\large\strut,
  coltitle=problemblue,
  colbacktitle=problemblue,
  title style={
left color=exercisebgblue,
    right color=white,
    middle color=exercisebgblue  
  },
  overlay={
    \draw[line width=1pt,problemblue] (frame.south west) -- (frame.south east);
    \draw[line width=1pt,problemblue] (frame.north west) -- (frame.north east);
    \draw[line width=1pt,problemblue] (frame.south west) -- (frame.north west);
    \draw[line width=1pt,problemblue] (frame.south east) -- (frame.north east);
  }
}
%----------------------------------------------------
\newtcolorbox[auto counter]{Activite}{
  % breakable,
  enhanced,
  colback=white,
  boxrule=0pt,
  arc=0pt,
  outer arc=0pt,
  title=Activité ~\thetcbcounter,
  fonttitle=\bfseries\sffamily\large\strut,
  coltitle=problemblue,
  colbacktitle=problemblue,
  title style={
left color=yellow!50!white,
    right color=white,
    middle color=yellow!20!white  
  },
  overlay={
    \draw[line width=1pt,problemblue] (frame.south west) -- (frame.south east);
    \draw[line width=1pt,problemblue] (frame.north west) -- (frame.north east);
    \draw[line width=1pt,problemblue] (frame.south west) -- (frame.north west);
    \draw[line width=1pt,problemblue] (frame.south east) -- (frame.north east);
  }
}
%----------------------------------------------------
\newtcolorbox[auto counter]{Application}{
  % breakable,
  enhanced,
  colback=white,
  boxrule=0pt,
  arc=0pt,
  outer arc=0pt,
  title=Application ~\thetcbcounter,
  fonttitle=\bfseries\sffamily\large\strut,
  coltitle=problemblue,
  colbacktitle=problemblue,
  title style={
left color=exercisebgblue,
    right color=white,
    middle color=exercisebgblue  
  },
  overlay={
    \draw[line width=1pt,problemblue] (frame.south west) -- (frame.south east);
    \draw[line width=1pt,problemblue] (frame.north west) -- (frame.north east);
    \draw[line width=1pt,problemblue] (frame.south west) -- (frame.north west);
    \draw[line width=1pt,problemblue] (frame.south east) -- (frame.north east);
  }
}
%----------------------------------------------------
\newtcolorbox{mybox}[2]{enhanced,breakable,
    before skip=2mm,after skip=2mm,
    colback=white,colframe=#2!30!blue,boxrule=0.3mm,rightrule=0.3mm,
    attach boxed title to top center={xshift=0cm,yshift*=1mm-\tcboxedtitleheight},
    varwidth boxed title*=-3cm,
    boxed title style={frame code={
    \path[fill=#2!30!black]
    ([yshift=-1mm,xshift=-1mm]frame.north west)
    arc[start angle=0,end angle=180,radius=1mm]
    ([yshift=-1mm,xshift=1mm]frame.north east)
    arc[start angle=180,end angle=0,radius=1mm];
    \path[draw=black,line width=1pt,left color=#2!1!white,right color=#2!1!blue!65,
    middle color=#2!1!green]
    ([xshift=-2mm]frame.north west) -- ([xshift=2mm]frame.north east)
    [rounded corners=1mm]-- ([xshift=1mm,yshift=-1mm]frame.north east)
    -- (frame.south east) -- (frame.south west)
    -- ([xshift=-1mm,yshift=-1mm]frame.north west)
    [sharp corners]-- cycle;
    },interior engine=empty,
    },
title=#1,coltitle=black,fonttitle=\sffamily}
%---------------------------------------------
\newtcolorbox{boxone}{%
    enhanced,
    colback=brown!10,
    boxrule=0pt,
    sharp corners,
    drop lifted shadow,
    frame hidden,
    fontupper=\bfseries,
    notitle,
    overlay={%
        \draw[Circle-Circle, brown!70!black, line width=2pt](frame.north west)--(frame.south west); 
        \draw[Circle-Circle, brown!70!black, line width=2pt](frame.north east)--(frame.south east);}
    }
\def\N{\mathbb{N}}
\def\Z{\mathbb{Z}}
\def\D{\mathbb{D}}
\def\Q{\mathbb{Q}}
\def\R{\mathbb{R}}

    
\begin{document}


\begin{tcolorbox}[title=\textcolor{blue}{\shadowbox{ Prof : Othmane Laksoumi}}
\hfill
\textcolor{blue}{\shadowbox{La droite dans le plan (Etude analytique)}}]
\end{tcolorbox}

\begin{mybox}{Lycée Qualifiant Zitoun}{gray}
    \begin{minipage}{8cm}
    \textcolor{darkbrown}{Année scolaire : } 2024-2025 \\
    \textcolor{darkbrown}{Niveau : } Tronc commun scientifique \\
    \textcolor{darkbrown}{Durée totale : } $5h$
    \end{minipage}
\end{mybox}

\begin{boxone}
{\Large\ding{45}}
\textcolor{red}{\large Contenus du programme :}
\begin{itemize}
     \item Le repère : coordonnées d’un point, coordonnées d’un vecteur;
    \item Condition de colinéarité de deux vecteurs;
    \item Détermination d’une droite définie par la donnée d’un point et d’un vecteur directeur;  
    \item Représentation paramétrique d’une droite;
    \item Équation cartésienne d’une droite;
    \item Positions relatives de deux droites.
\end{itemize}

{\Large\ding{45}}
\textcolor{red}{\large Les capacités attendues :}
\begin{itemize}
    \item Exprimer les notions et les propriétés de la géométrie affine et de la géométrie vectorielle à l’aide des coordonnées.
    \item Utiliser l’outil analytique dans la résolution de problèmes géométriques.
\end{itemize}

{\Large\ding{45}}
\textcolor{red}{\large Recommandations pédagogiques :} 
  \begin{itemize}
      \item Il faudra habituer les élèves à l’utilisation des différentes méthodes pour exprimer la colinéarité de deux vecteurs.
  \end{itemize}
\end{boxone}

\newpage

\begin{tabular}{|>{\raggedright\arraybackslash}p{17cm}|>{\centering\arraybackslash}p{0.8cm}|}
\hline
\rowcolor{head}

\centering Contenu du cours &
 Durée \\
\hline

\vspace{1mm}
\mysection{Repère : Coordonnées d'un point - Coordonnées d'un vecteur}

\mysubsection{1}{Repère}
\begin{Definition}
Soit $(O;I;J)$ un repère du plan.\\
On pose $\vec{i} = \overrightarrow{OI}$ et $\vec{j} = \overrightarrow{OJ}$.
\begin{itemize}
	\item Le triplet $(O;I;J)$ s'appelle repère du plan.
	\item $(O;I;J)$ est un repère orthogonal si $(OI)\perp (OJ)$.
	\item $(O;I;J)$ est un repère orthonormal (ou orthonormé) si $\vert\vert\vec{i}\vert\vert = \vert\vert\vec{j}\vert\vert = 1$ (c'est-à-dire $OI=OJ=1$).
	\item La droite $(OI)$ est appelée axe des abscisses.
	\item La droite $(OJ)$ est appelée axe des ordonnées.
\end{itemize}
\end{Definition}
\mysubsection{2}{Coordonnées d'un point - Coordonnées d'un vecteur}
\begin{Proposition}
Soit $(O;\vec{i};\vec{j})$ un repère du plan.
\begin{itemize}
	\item Pour tout point $M$ du plan, il existe un couple unique $(x;y)$ de nombres réels tel que :
	$$\overrightarrow{OM} = x\vec{i} + y\vec{j}$$
	Le couple $(x;y)$ est appelé couple de coordonnées du point $M$. On écrit $M(x;y)$.
	\item Pour tout vecteur $\vec{u}$ du plan, il existe un couple unique $(x;y)$ de nombres réels tel que : $$\vec{u} = x\vec{i} + y\vec{j}$$
	Le couple $(x;y)$ est appelé couple de coordonnées du vecteur $M\vec{u}$. On écrit $u(x;y)$ ou $\vec{u}\left(\begin{matrix}
x\\
y
\end{matrix}\right)$.
\end{itemize}
\end{Proposition}
\begin{Proposition}
\begin{itemize}
	\item \textcolor{red}{Egalité de deux vecteurs} : 
	Soient $\vec{u}(x;y)$ et $\vec{v}(x^{\prime};y^{\prime})$ deux vecteurs.\\
	$\vec{u}(x;y)$ = $\vec{v}(x;y)$ signifie que $x = x^{\prime}$ et $y = y^{\prime}$
	\item \textcolor{red}{Coordonnées du vecteur $\overrightarrow{AB}$} : $\overrightarrow{AB}(x_B - x_A; y_B - y_A)$.	
	\item \textcolor{red}{Coordonnées de la somme de deux vecteurs} : $(\vec{u} + \vec{v})(x+x^{\prime}; y + y^{\prime})$.
	\item \textcolor{red}{Coordonnées du produit d'un vecteur par un réel} : $k\vec{u}(kx,ky)$ où $k\in\mathbb{R}$.
	\item \textcolor{red}{Colinéarité de deux vecteurs non nuls} :\\
	$\vec{u}$ et $\vec{v}$ sont colinéaires si et seulement s'il existe un réel $k$ tel que : $x^{\prime} = kx$ et $y^{\prime} = ky$.
	\item \textcolor{red}{Milieu d'un segment} : Si $I$ est le milieu du segment $[AB]$, alors $I\left(\dfrac{x_A + x_B}{2};\dfrac{y_A + y_B}{2}\right)$.
	\item \textcolor{red}{Distance de deux points} : Dans un repère orthonormal, $AB = \sqrt{(x_B - x_A)^2 + (y_B - y_B)^2}$
\end{itemize}
\end{Proposition}

&\\
\hline
\end{tabular}

\begin{tabular}{|>{\raggedright\arraybackslash}p{17cm}|>{\centering\arraybackslash}p{0.8cm}|}
\hline
\vspace{1mm}
\mysection{Condition de colinéarité de deux vecteurs}
\begin{Definition}
On considére deux vecteurs $\vec{u}(x;y)$ et $\vec{v}(x^{\prime};y^{\prime})$.
Le nombre $xy^{\prime} - yx^{\prime}$ s'appelle déterminant des deux vecteurs $\vec{u}$ et $\vec{v}$ et se note $det(\vec{u};\vec{v})$. On écrit $det(\vec{u};\vec{v}) = \begin{array}{|c c|}
x & x^{\prime}\\
y & y^{\prime}
\end{array} = xy^{\prime} - yx^{\prime}$.
\end{Definition}
\begin{Exemple}
On considére les vecteurs $\vec{u} = 3\vec{i} - 4\vec{j}$ et $\vec{v} = \vec{i} + 2\vec{j}$.\\
On a $det(\vec{u};\vec{v}) = \begin{array}{|c c|}
3 & 1\\
-4 & 2\end{array} = 3\times 2 - 1\times (-4) = 6 + 4 = 10$
\end{Exemple}
\begin{Proposition}
\begin{itemize}
	\item $\vec{u}$ et $\vec{v}$ sont colinéaires si et seulement si $det(\vec{u};\vec{v}) = 0$.
	\item $\vec{u}$ et $\vec{v}$ sont non colinéaires si et seulement si $det(\vec{u};\vec{v}) \neq 0$.
	\item Les points $A,\ B$ et $C$ sont alignés si et seulement si $det(\overrightarrow{AB};\overrightarrow{AC}) = 0$.
\end{itemize}
\end{Proposition}
\begin{Exemple}
On considére les vecteurs $\vec{u} = \vec{i} - 2\vec{j}$ et $\vec{v} = 2\vec{i} - 4\vec{j}$.\\
On considére les points $A(-1;2),\ B(2;3)$ et $C\left(\dfrac{1}{2};\dfrac{3}{2}\right)$.\\
\end{Exemple}
\mysection{La droite dans le plan}
\begin{Definition}
Soit $A$ un point du plan et soit $\vec{u}$ un vecteur non nul.\\
L'ensemble des points $M$ qui vérifient : $\overrightarrow{AM} = k\vec{u}$, où $k\in\mathbb{R}$, est la droite passant par le point $A$ et dirigée par le vecteur $\vec{u}$. On la note $D(A,\vec{u})$.
\end{Definition}
\begin{Exemple}
Construire la droite $D(A;\vec{u})$, avec $A(1;2)$ et $\vec{u}(-1;1)$.
\end{Exemple}
\mysection{Représentation paramétrique d'une droite}
\begin{Definition}
Le plan $\mathcal{P}$ est rapporté à un repère $(O; \vec{i}, \vec{j})$.

Soient $A(x_0; y_0)$ un point du plan et $\vec{u}(\alpha; \beta)$ un vecteur non nul.\\

$\text{Le système}
\begin{cases}
x = x_0 + k \alpha \\
y = y_0 + k \beta
\end{cases} \quad (k \in \mathbb{R})
$\\
s'appelle \textcolor{red}{représentation paramétrique} de la droite $(D)$ passant par le point $A(x_0; y_0)$ et dirigée par le vecteur $\vec{u}(\alpha; \beta)$.
\end{Definition}
\textcolor{red}{Remarque :}

La droite $(AB)$ est dirigée par le vecteur $\overrightarrow{AB}$.
&\\
\hline
\end{tabular}

\begin{tabular}{|>{\raggedright\arraybackslash}p{17cm}|>{\centering\arraybackslash}p{0.8cm}|}
\hline
\vspace{1mm}
\begin{Exemple}
\begin{itemize}
    \item Soit $(D)$ la droite passant par le point $A(2; -3)$ et dirigée par le vecteur $\vec{u} = (3; -5)$.\\
    $(D)$ est définie par la représentation paramétrique suivante :
    \[
    \begin{cases}
    x = 2 + 3k \\
    y = -3 - 5k
    \end{cases} \quad (k \in \mathbb{R})
    \]

    Le nombre $k$ est le paramètre dépendant du point $M$.
    \begin{itemize}
        \item Si on prend $k = 0$, on obtient le point $A(2; -3)$ qui est évidemment un point de $(D)$.
        \item Si on prend $k = 1$, on obtient le point $B(5; -8)$ appartenant à $(D)$.
        \item Si on prend $k = \frac{1}{2}$, on obtient le point $C\left( \frac{7}{2}; -\frac{11}{2} \right)$ qui appartient donc à la droite $(D)$.
    \end{itemize}

    \item Le système
    \[
    \begin{cases}
    x = 1 + 2t \\
    y = 1 - 4t
    \end{cases} \quad (t \in \mathbb{R})
    \]
    détermine une représentation paramétrique de la droite $(E)$ de point $A(1; 1)$ et de vecteur directeur $\vec{v}(2; -4)$.
\end{itemize}
\end{Exemple}
\begin{Application}
On considère les points $E(-1; 2)$, $F(2; 3)$ et $G(1;-1)$.
    \begin{enumerate}
        \item Déterminer une représentation paramétrique de la droite $(EF)$.
        \item Le point $G$ appartient-il à la droite $(EF)$?
        \item Déterminer une représentation paramétrique de la droite $G$ passant par le point $L$ et dirigée par le vecteur $\vec{w} = 3 \vec{i} + \vec{j}$.
    \end{enumerate}
\end{Application}

\mysection{Equation cartésienne d'une droite}
\begin{Definition}
Le plan $\mathcal{P}$ est rapporté à un repère $(O; \vec{i}, \vec{j})$.\\
Toute droite, dans le plan, admet une équation cartésienne de la forme : 
$$ax + by + c = 0 \text{ où } (a;b)\neq (0;0)$$
\end{Definition}
\begin{Exemple}
Déterminons une équation cartésienne de la droite $(D)$ passant par $A(-1;2)$ et de vecteur directeur $\vec{u}(-3;4)$.
\end{Exemple}
\begin{Application}
On considére les deux points $B(1;-1)$ et $C(-2;3)$.\\
Déterminer une équation cartésienne de la droite $(BC)$.
\end{Application}
\begin{Proposition}
%Le plan $\mathcal{P}$ est rapporté à un repère $(O; \vec{i}, \vec{j})$.\\
Soient $a,\ b$ et $c$ des nombres réels avec $(a;b)\neq (0;0)$.\\
L'ensemble des points $M(x;y)$ tels que $ax + by + c = 0$ est une droite dirigée par le vecteur $\vec{u}(-b;a)$.
\end{Proposition}
&\\
\hline

\end{tabular}

\begin{tabular}{|>{\raggedright\arraybackslash}p{17cm}|>{\centering\arraybackslash}p{0.8cm}|}
\hline
\vspace{1mm}
\begin{Exemple}
Soit $(\Delta)$ la droite d'équation : $2x - y + 1 = 0$.
La droite $(\Delta)$ est la droite passant par $A(-1;1)$ et de vecteur directeur $\vec{u}(1;-2)$.
\end{Exemple}
\begin{Application}
Soit $(L)$ la droite d'équation : $3x + 4y - 5 = 0$.
\begin{enumerate}
	\item Déterminer un point appartenant à la droite $(L)$.
	\item Déterminer un directeur de la droite $(L)$.
	\item Le point $B\left(\dfrac{1}{3};\dfrac{5}{2}\right)$ apparitent-il à $(L)$?
\end{enumerate}
\end{Application}
\mysection{Droites particuliéres}

\mysubsection{1}{Droite paralléle à l'axe des ordonnées}
\begin{Proposition}
Le plan $\mathcal{P}$ est rapporté à un repère $(O,\vec{i};\vec{j})$.\\
Pour qu'une droite soit paralléle à l'axe des ordonnées, il faut et il suffit qu'elle ait une équation cartésienne de la forme : $x = c$.
\end{Proposition}
\mysubsection{2}{Droite paralléle à l'axe des abscisses}
\begin{Proposition}
Le plan $\mathcal{P}$ est rapporté à un repère $(O,\vec{i};\vec{j})$.\\
Pour qu'une droite soit paralléle à l'axe des abscisses, il faut et il suffit qu'elle ait une équation cartésienne de la forme : $y = c$.
\end{Proposition}
\begin{Exemple}
Construire les droites d'équation $x = 1$ et $y = -2$.
\end{Exemple}
\mysection{Équation d'une droite et son coefficient directeur}
\begin{Proposition}
Le plan $\mathcal{P}$ est rapporté à un repère $(O,\vec{i};\vec{j})$.\\
Pour qu'une droite $(D)$ ne soit pas paralléle à l'axe des ordonnées, il faut et il suffit qu'elle ait une équation de la forme : $y = mx + p$.
\begin{itemize}
	\item Le nombre réel $m$ s'appelle \textcolor{red}{coefficient directeur} de la droite $(D)$.
	\item Le nombre réel $p$ s'appelle \textcolor{red}{l'ordonnée à l'origine} de $(D)$.
	\item L'équation $y = mx + p$ est \textcolor{red}{l'équation réduite} de la droite $(D)$.
\end{itemize}
\end{Proposition}
\begin{Proposition}
Le plan $\mathcal{P}$ est rapporté à un repère $(O,\vec{i};\vec{j})$.\\
La droite passant par le point $A(x_0;y_0)$ et de coefficient directeur $m$ a une équation cartésienne de la forme : $y - y_0 = m(x - x_0)$.
\end{Proposition}
&\\
\hline
\end{tabular}

\begin{tabular}{|>{\raggedright\arraybackslash}p{17cm}|>{\centering\arraybackslash}p{0.8cm}|}
\hline
\vspace{1mm}
\begin{Exemple}
Déterminons une équation de la droite passant par $A(-1;2)$ et de coefficient directeur $2$.
\end{Exemple}
\mysection{Positions relatives de deux droites}

\mysubsection{1}{Parallélisme de deux droites définies par leurs équations cartésiennes}
\begin{Theoreme}
Le plan $\mathcal{P}$ est rapporté à un repère $(O,\vec{i};\vec{j})$.\\
Pour que deux droites d'équation cartésienne respectives $ax + by + c = 0$ et $a'x + b^{\prime}y + c' = 0$ (où $(a;b)\neq (0;0)$ et $(a';b')\neq (0;0)$) soient parallèles, il faut et il suffit que : $ab' - ba' = 0$ c'est-à-dire $\begin{array}{|cc|}
a & a' \\
b & b'
\end{array} = 0$.
\end{Theoreme}
\begin{Exemple}
On considére les deux droites $\begin{cases}
(D) : 2x - y + 3 = 0\\
(D') : 3x + 4y -1 = 0
\end{cases}$
\end{Exemple}
\mysubsection{2}{Parallélisme de deux droites définies par leurs équations réduites}
\begin{Theoreme}
Le plan $\mathcal{P}$ est rapporté à un repère $(O,\vec{i};\vec{j})$.\\
Pour que deux droites d'équations respectives $y = mx + p$ et $y = m'x + p'$ soient paralléles, il faut et il suffit que $m = m'$.
\end{Theoreme}
\begin{Exemple}
On considére la droite $(D)$ d'équation : $y = -\dfrac{1}{2}x + 3$.
Déterminons une équation de la droite $(\Delta)$ passant par $A(3;-1)$
\end{Exemple}
\mysubsection{3}{Intersection de deux droites}
\begin{Theoreme}
Le plan $\mathcal{P}$ est rapporté à un repère $(O,\vec{i};\vec{j})$.\\
\begin{itemize}
	\item Pour que les droites $(D)$ et $(D')$ d'équations respectives : 
	$ax + by + c = 0 \quad\text{et}\quad a'x + b'y + c' = 0$
	soient sécantes, il faut et il suffit que $ab' - ba' \neq 0$.\\
	Le couple de coordonnées du point d'intersection de $(D)$ et $^(D')$ est la solution du système : $\begin{cases}
	ax + by + c = 0 \\
	a'x + b'y + c' = 0
	\end{cases}$
	\item Pour que les droites $(\Delta)$ et $(\Delta')$ d'équations respectives : 
	$y = mx + p \quad\text{et}\quad y = m'x + p'$
	soient sécantes, il faut et il suffit que $m\neq m'$.\\
	Le couple de coordonnées du point d'intersection de $(\Delta)$ et $(\Delta')$ est la solution du système : $\begin{cases}
	y = mx + p \\
	y = m'x + p'
	\end{cases}$
\end{itemize}
\end{Theoreme}
&\\
\hline
\end{tabular}
\newpage
\begin{tabular}{|>{\raggedright\arraybackslash}p{17cm}|>{\centering\arraybackslash}p{0.8cm}|}
\hline
\vspace{1mm}
\begin{Exemple}
Étudions l'intersection des droites $(D)$ et $(D')$ d'équations respectives :
$$2x + y + 1 = 0 \quad\text{et}\quad 3x + 4y - 5 = 0$$
On a : $\begin{array}{|cc|}
2 & 3\\
1 & 4
\end{array} = 8 - 3 = 5$
Puisque $5\neq 0$, les droites $(D)$ et $(D')$ sont sécantes; le couple de coordonnées du point $I$ d'intersection de $(D)$ et $(D')$ est la solution du système :
$$\begin{cases}
2x + y + 1 = 0\\
3x + 4y - 5 = 0
\end{cases}
\quad\text{c'est-à-dire}\quad
\begin{cases}
2x + y = -1\\
3x + 4y = 5
\end{cases}$$
La solution de ce système est $(x;y)$ tel que :
$$x = \dfrac{\begin{array}{|cc|} -1 & 5 \\ 
1 & 4\end{array}}{5} = -\dfrac{9}{5}\quad\text{et}\quad y = \dfrac{\begin{array}{|cc|} 2 & -1 \\ 
3 & 5\end{array}}{5} = \dfrac{13}{5}$$
\end{Exemple}

&\\
\hline
\end{tabular}





























\end{document}