\documentclass[12pt,a4paper]{article}
\usepackage[top=2cm,left=1cm,right=1cm,bottom=2cm]{geometry}
\usepackage{tikz}
 \usepackage[tikz]{bclogo}%
 \usetikzlibrary{calc,intersections}
\usetikzlibrary{arrows.meta}
\usetikzlibrary{decorations.pathmorphing}
\usepackage{amssymb,mathtools,amsthm}
\usepackage{fourier}
\usepackage{tkz-tab}
\usepackage{xcolor}
\usepackage{multicol, array, fancyhdr}
\usepackage{tasks}
\newcommand{\Lim}{\displaystyle\lim}
\renewcommand{\columnseprule}{1pt}
\renewcommand{\arraystretch}{1.5}
\usepackage{tikz}
\usetikzlibrary{positioning}

% \renewcommand{\displaystyle\frac}[2]{\displaystyle\displaystyle\frac{#1}{#2}}


%======================================================
\newtheoremstyle{mystyle}
  {\topsep}% espace avant
  {\topsep}% espace après
  {\upshape}% police du corps du théorème
  {}% indentation (vide pour rien, \parindent)
  {\bfseries\sffamily}% police du titre du théorème
  { :\newline}% ponctuation après le théorème
  { }% après le titre du théorème (espace ou \newline)
  {%
    \rule[0.5\baselineskip]{0.5\textwidth}{1pt}%
    \newline\fcolorbox{black}{white}{%
      \thmname{#1}\thmnumber{ \textup{#2}}\thmnote{ \textnormal{(#3)}}%
    }%
    
    % \vspace{0.5em} % Adjust vertical space after the title
  }% spécifications du titre

\theoremstyle{mystyle}
\newtheorem{exo}{Exercice}

%======================================================



\begin{document}


\pagestyle{fancy}
\fancyhf{} % clear all header and footer fields
\fancyhead[L]{Lycée : Zitoun \hspace{1.5cm} Année scolaire : 2024-2025} % Left header
\fancyhead[C]{ \hspace{4cm} Niveau : 2BAC PC} % Right-Center header
\fancyhead[R]{Prof. Othmane Laksoumi} % Right header
\fancyfoot[C]{\thepage} % Footer


\begin{center}
    \textbf{\Large Résumé : Dérivation}
\end{center}

\subsection*{Déterminant de deux vecteurs}

Le déterminant de deux vecteurs $\vec{u}(x; y)$ et $\vec{v}(x'; y')$ est le nombre réel noté :
\[
\det(\vec{u}, \vec{v}) = 
\begin{vmatrix}
x & x' \\
y & y'
\end{vmatrix} = x \cdot y' - y \cdot x'.
\]

\subsection*{Condition de colinéarité de deux vecteurs}

\begin{itemize}
    \item $\vec{u}$ et $\vec{v}$ sont colinéaires si et seulement si $\det(\vec{u}, \vec{v}) = 0$.
    \item $\vec{u}$ et $\vec{v}$ ne sont pas colinéaires si et seulement si $\det(\vec{u}, \vec{v}) \neq 0$.
\end{itemize}





















\end{document}