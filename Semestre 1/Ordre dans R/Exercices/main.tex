\documentclass[12pt,a4paper]{article}
\usepackage[top=2cm,left=1cm,right=1cm,bottom=2cm]{geometry}
\usepackage{amssymb,mathtools,amsthm}
\usepackage{fourier}
\usepackage{xcolor}
\usepackage{multicol, array, fancyhdr}
\usepackage{tasks}
\newcommand{\Lim}{\displaystyle\lim}
\renewcommand{\columnseprule}{1pt}
\renewcommand{\arraystretch}{1.5}
% \renewcommand{\frac}[2]{\displaystyle\frac{#1}{#2}}


%======================================================
\newtheoremstyle{mystyle}
  {\topsep}% espace avant
  {\topsep}% espace après
  {\upshape}% police du corps du théorème
  {}% indentation (vide pour rien, \parindent)
  {\bfseries\sffamily}% police du titre du théorème
  { :}% ponctuation après le théorème
  { }% après le titre du théorème (espace ou \newline)
  {%
    \rule[0.5\baselineskip]{0.5\textwidth}{1pt}%
    \newline\fcolorbox{black}{white}{%
      \thmname{#1}\thmnumber{ \textup{#2}}\thmnote{ \textnormal{(#3)}}%
    }%
    
    % \vspace{0.5em} % Adjust vertical space after the title
  }% spécifications du titre

\theoremstyle{mystyle}
\newtheorem{exo}{Exercice}

%======================================================



\begin{document}


\pagestyle{fancy}
\fancyhf{} % clear all header and footer fields
\fancyhead[L]{Lycée : Zitoun \hspace{1.5cm} Année scolaire : 2024-2025} % Left header
\fancyhead[C]{ \hspace{4cm} Niveau : TCS} % Right-Center header
\fancyhead[R]{Prof. Othmane Laksoumi} % Right header
\fancyfoot[C]{\thepage} % Footer


\begin{center}
    \textbf{\Large L'ordre dans $\mathbb{R}$ }
\end{center}
\begin{multicols*}{2}
\begin{exo}
Comparer les nombres réels $a$ et $b$ dans les cas suivants :
\begin{enumerate}
	\item $a = \dfrac{496}{55} \ ; \ b = 9$
	\item $a = \dfrac{386}{35} \ ; \ b = 11$
	\item $a = 2\sqrt{3} \ ; \ b = 3\sqrt{5}$
    \item $a = 5 + \sqrt{2} \ ; \ b = \sqrt{25 + 10 \sqrt{2}}$
    \item $a = \sqrt{11} \ ; \ b = \sqrt{5 + \sqrt{6}}$
    \item $a = 2\sqrt{5} - 3\sqrt{2} \ ; \ b = \sqrt{39 - 12\sqrt{10}}$
    \item $a = \dfrac{\sqrt{3} -1}{\sqrt{3}} + 1 \ ; \ b = \dfrac{2-\sqrt{3}}{\sqrt{3} - 1}$
	\item $a = x\sqrt{x+1} \ ; \ b = (x+1)\sqrt{x}\quad x>0$  
    \item $a = (x + y)^2 \ ; \ b = 4xy \quad (x, y \in \mathbb{R})$
\end{enumerate}
\end{exo}

\begin{exo}
\text{ }
\begin{enumerate}
    \item Écrire sans symbole de la valeur absolue les nombres suivants :  
    $$|3 - 2\sqrt{3}|, \quad |2\sqrt{3} - 3\sqrt{2}|, \quad | \sqrt{2} - 2|, \quad \sqrt{(5 - 2\sqrt{2})^2}$$
    

    \item Résoudre dans \(\mathbb{R}\) :
    $$
    \hspace{-0.6cm}|4x - 5| = 1, \quad |3x + 7| \leq 2, \quad |4x - 9| > 1, \quad |3x + 7| = 2
    $$
  	$$
    |2x+1| = |x+5|, \quad |x-7| = 0, \quad |x+5| < 4.
    $$
\end{enumerate}

\end{exo}

\begin{exo}
Soit \((x, y) \in \mathbb{R}^2\) tels que \( x \geq \dfrac{1}{2} \), \( y \leq 1 \), et \( x - y = 3 \).  
   	\begin{enumerate}
        \item Simplifier le nombre \( E \) tel que : 
        $$E = \sqrt{(2x - 1)^2} + \sqrt{(2y - 2)^2}.$$
        \item Vérifier que \( \dfrac{1}{2} \leq x \leq 4 \) et \( \dfrac{-5}{2} \leq y \leq 1 \).
        \item Calculer la valeur de \( F \) tel que :  
        \[
        F = |x+y-5| + |x+y+2|.
        \]
	\end{enumerate}
\end{exo}

\begin{exo}
\text{ }
\begin{enumerate}
    \item Écrire sous forme d'un intervalle les ensembles suivants :\\
         $x \leq -3$ ; $x \geq -1$ ; $x \leq \sqrt{2}$ ; $x > -5$ ; $-4 \leq x < 6$

    \item Écrire si possible sous forme d'un intervalle les ensembles suivants :
    \begin{enumerate}
        \item $[-3;3] \cap ]0;5[$ et $[-3;3] \cup ]0;5[$
        \item $[0;5] \cap ]4;+\infty[$ et $[0;5] \cup ]4;+\infty[$
        \item $]-\infty;1] \cap [-3;3[$ et $]-\infty;1] \cup [-3;3[$
    \end{enumerate}
\end{enumerate}
\end{exo}

\begin{exo}
\text{ }
\begin{enumerate}
    \item Soit $a$ et $b$ deux réels tels que \\
    $3 \leq a \leq 4$ et $2 \leq b \leq 7$
 	
 	Donner l'encadrement de $a + b$, $a - b$, $a \times b$, et $\dfrac{a}{b}$.
    \item Soit $a$ et $b$ deux réels tels que \\
    $-3 \leq a \leq 4$ et $2 \leq b \leq 7$
    
    Donner l'encadrement de $a + b$, $a - b$, $a \times b$, et $\dfrac{a}{b}$.
    \item Soit $x$ et $y$ deux réels tels que\\
     $2 \leq x \leq 5$ et $-4 \leq y \leq 1$ \\
    Donner un encadrement de $-x + 2y$, $xy - 4$, $\dfrac{x}{y+5}$, $\dfrac{x + y + 3}{x - y}$.
\end{enumerate}
\end{exo}

\begin{exo}
Montrer, dans chaque cas, que $A$ est une valeur approchée du nombre $x$ à $r$ prés :
\begin{enumerate}
	\item $x = \dfrac{7}{6} \quad;\quad A = 1,1666 \quad;\quad r = 10^{-4}$
	\item $x = \dfrac{1}{3} \quad;\quad A = 0,4 \quad;\quad r = 8\times 10^{-2}$
\end{enumerate}
\end{exo}

\begin{exo}
Soit $a$ une approximation de $\frac{1}{2}$ à la précision $\dfrac{1}{12}$ près.
\begin{enumerate}
    \item Montrer que $\frac{5}{12} < a < \frac{7}{12}$.
    \item Encadrer le nombre $\dfrac{a}{3a - 1}$.
    \item En déduire que $\frac{13}{9}$ est une approximation de $\dfrac{a}{3a - 1}$ à la précision $\dfrac{8}{9}$ près.
\end{enumerate}
\end{exo}



















\end{multicols*}



\end{document}
