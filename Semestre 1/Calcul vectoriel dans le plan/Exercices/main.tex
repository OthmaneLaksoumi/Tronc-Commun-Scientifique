\documentclass[12pt,a4paper]{article}
\usepackage[top=2cm,left=1cm,right=1cm,bottom=2cm]{geometry}
\usepackage{amssymb,mathtools,amsthm}
\usepackage{fourier}
\usepackage{xcolor}
\usepackage{multicol, array, fancyhdr}
\usepackage{tasks}
\newcommand{\Lim}{\displaystyle\lim}
\renewcommand{\columnseprule}{1pt}
\renewcommand{\arraystretch}{1.5}
% \renewcommand{\frac}[2]{\displaystyle\frac{#1}{#2}}


%======================================================
\newtheoremstyle{mystyle}
  {\topsep}% espace avant
  {\topsep}% espace après
  {\upshape}% police du corps du théorème
  {}% indentation (vide pour rien, \parindent)
  {\bfseries\sffamily}% police du titre du théorème
  { :\newline}% ponctuation après le théorème
  { }% après le titre du théorème (espace ou \newline)
  {%
    \rule[0.5\baselineskip]{0.5\textwidth}{1pt}%
    \newline\fcolorbox{black}{white}{%
      \thmname{#1}\thmnumber{ \textup{#2}}\thmnote{ \textnormal{(#3)}}%
    }%
    
    % \vspace{0.5em} % Adjust vertical space after the title
  }% spécifications du titre

\theoremstyle{mystyle}
\newtheorem{exo}{Exercice}

%======================================================



\begin{document}


\pagestyle{fancy}
\fancyhf{} % clear all header and footer fields
\fancyhead[L]{Lycée : Zitoun \hspace{1.5cm} Année scolaire : 2024-2025} % Left header
\fancyhead[C]{ \hspace{4cm} Niveau : TCS} % Right-Center header
\fancyhead[R]{Prof. Othmane Laksoumi} % Right header
\fancyfoot[C]{\thepage} % Footer


\begin{center}
    \textbf{\Large Calcul vectoriel dans le plan }
\end{center}
\begin{multicols*}{2}

\begin{exo}
   Soit $ABCD$ un parallélogramme.

   Simplifier les écritures vectorielles suivantes :
   $\overrightarrow{CD} + \overrightarrow{DA}\quad;\quad \overrightarrow{CB} + \overrightarrow{CD}\quad;\quad \overrightarrow{AB} + \overrightarrow{AD}\quad;\quad \overrightarrow{AC} + \overrightarrow{BA} + \overrightarrow{CB}\quad;\quad \overrightarrow{CD} + \overrightarrow{AC} + \overrightarrow{DA}$
\end{exo}

\begin{exo}
    Montrer que $\overrightarrow{AB} = \overrightarrow{DC}$ signifie que $\overrightarrow{AD} = \overrightarrow{BC}$.
\end{exo}

\begin{exo}
Soient $A$, $B$, $C$ et $D$ quatre points distincts.
On pose $\overrightarrow{u} = \overrightarrow{AB}$ et $\overrightarrow{v} = \overrightarrow{CD}$.

Construire les vecteurs :

$2\overrightarrow{u}\quad;\quad \displaystyle\frac{1}{3}\overrightarrow{u}\quad;\quad -\overrightarrow{u}\quad;\quad -3\overrightarrow{u}\quad;\quad\displaystyle\frac{3}{4}\overrightarrow{u}\quad;\quad\displaystyle\frac{6}{5}\overrightarrow{u}$

$\overrightarrow{u} + \overrightarrow{v}\quad;\quad \overrightarrow{u} - \overrightarrow{v}\quad;\quad 2\overrightarrow{u} + \overrightarrow{v}\quad;\quad\displaystyle\frac{1}{3}\overrightarrow{u} + \displaystyle\frac{1}{5}\overrightarrow{v}$
\end{exo}

\begin{exo}
Soit $ABCD$ un parallélogramme de centre $O$.

Montrer que tous ces vecteurs sont nuls :
\begin{enumerate}
    \item $\overrightarrow{u} = \overrightarrow{OA} + \overrightarrow{OB} + \overrightarrow{OC} + \overrightarrow{OD}$
    \item $\overrightarrow{v} = \overrightarrow{AO} - \overrightarrow{BO} + \overrightarrow{CO} - \overrightarrow{DO} $
    \item $\overrightarrow{w} = \overrightarrow{AB} + 2\overrightarrow{BC} -\overrightarrow{AC} + \overrightarrow{AD} $
\end{enumerate}
\end{exo}

\begin{exo}
   Soient $A$, $B$, $C$, $M$ quatre points du plan et soit $\overrightarrow{u}$ le vecteur défini par :
$$
\overrightarrow{u} = \overrightarrow{MA} + 2\overrightarrow{MB} - 3\overrightarrow{MC}
$$
\begin{enumerate}
    \item Montrer que : $\overrightarrow{u} = 2\overrightarrow{AB} - 3\overrightarrow{AC}$
    \item Soit $\overrightarrow{v}$ le vecteur défini par : $\overrightarrow{v} = 2\overrightarrow{BA} - 6\overrightarrow{BC}$ \\
    Montrer que les vecteurs $\overrightarrow{u}$ et $\overrightarrow{v}$ sont colinéaires.
\end{enumerate}
\end{exo}
\columnbreak 
\begin{exo}
$ABCD$ est un parallélogramme et $M$ le point du plan tel que :
\[
\overrightarrow{AM} = \overrightarrow{AB} + 2\overrightarrow{AD}
\]
\begin{enumerate}
    \item Montrer que $B$, $C$, $M$ sont alignés.
    \item En déduire que $C$ est le milieu du segment $[BM]$.
\end{enumerate}
\end{exo}

\begin{exo}
$ABC$ est un triangle et $I$ et $J$ sont les milieux des segments $[AB]$ et $[AC]$ respectivement. \\
Montrer que :
\[
\overrightarrow{BC} = 2\overrightarrow{JI}
\]
\end{exo}

\begin{exo}
$ABC$ est un triangle. On considère $I$ et $J$ les milieux des segments $[AB]$ et $[AC]$ respectivement.
\begin{enumerate}
    \item Montrer que : $\overrightarrow{BJ} = -\frac{1}{2}\overrightarrow{AB} + \frac{1}{2}\overrightarrow{AC}$ et $\overrightarrow{CI} = \frac{1}{2}\overrightarrow{AB} + \frac{1}{2}\overrightarrow{AC}$.
    \item Soient $M$ et $N$ deux points du plan tels que : $\overrightarrow{BM} = 2\overrightarrow{BJ}$ et $\overrightarrow{CN} = 2\overrightarrow{CI}$.
    \begin{enumerate}
        \item Montrer que la nature du quadrilatère $ABCN$ est $ABCM$.
        \item Montrer que les points $A$, $M$, $N$ sont alignés.
    \end{enumerate}
\end{enumerate}
\end{exo}

\begin{exo}
    $ABCD$ est un parallélogramme et $M$ et $N$ deux points du plan tels que :
\[
\overrightarrow{BM} = \frac{1}{2}\overrightarrow{AB} \quad \text{et} \quad \overrightarrow{AN} = 3\overrightarrow{AD}
\]
\begin{enumerate}
    \item Construire une figure convenable.
    \item Montrer que : $\overrightarrow{CM} = \frac{1}{2}\overrightarrow{AB} - \overrightarrow{BC}$ et $\overrightarrow{CN} = 2\overrightarrow{AD} - \overrightarrow{DC}$.
    \item Montrer que $C$, $M$, et $N$ sont alignés.
    \item Soit $E$ le milieu de $[DN]$ et soit $F$ le point du plan tel que : $\overrightarrow{AB} = \overrightarrow{BF}$.
    \item Montrer que $C$ est le milieu de $[EF]$.
    \item Montrer que $ (\overrightarrow{BD})//(\overrightarrow{EF}) $.
\end{enumerate}

\end{exo}


\end{multicols*}



\end{document}
