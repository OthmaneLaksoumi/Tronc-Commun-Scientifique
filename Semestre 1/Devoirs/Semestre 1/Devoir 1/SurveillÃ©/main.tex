\documentclass[12pt,a4paper]{article}
\usepackage[top=2cm,left=1cm,right=1cm,bottom=2cm]{geometry}
\usepackage{tikz}
 \usepackage[tikz]{bclogo}%
 \usetikzlibrary{calc,intersections}
\usetikzlibrary{arrows.meta}
\usetikzlibrary{decorations.pathmorphing}
\usepackage{amssymb,mathtools,amsthm}
\usepackage{fourier}
\usepackage{tkz-tab}
\usepackage{xcolor}
\usepackage{multicol, array, fancyhdr}
\usepackage{tasks}
\newcommand{\Lim}{\displaystyle\lim}

\begin{document}


\pagestyle{fancy}
\fancyhf{} % clear all header and footer fields
\fancyhead[L]{Lycée : Zitoun \hspace{1.5cm} Année scolaire : 2024-2025} % Left header
\fancyhead[C]{ \hspace{4cm} Niveau : 2BAC PC} % Right-Center header
\fancyhead[R]{Prof. Othmane Laksoumi} % Right header
%\fancyfoot[C]{\thepage} % Footer


\begin{center}
    \textbf{\Large  Devoir Surveillé 1}
\end{center}

\underline{\large\textbf{Exercice 1 :}}(6 pt)
\begin{enumerate}
    % \item Déterminer les nombres pairs et les nombres impairs parmi les nombres suivants : (0.25 + 0.25 + 0.25 + 0.25 pt)
    \item Soit $n$ un entier naturel. Étudier la parité des nombres suivants : ( 2 pt)
    $$A = 14n + 28\quad;\quad B = 8n^3 + 4n^2 + 2n + 1 \quad;\quad C = (2n + 1)(6n + 21) \quad ; \quad D = 2^n + 3^n + 1$$
    % $$536\times 123\quad;\quad 33\times 1399\quad;\quad 15559 + 125699\quad;\quad 33369 - 4890$$
    \item Déterminer les nombres premiers parmi les nombres suivants :(1 pt)
    $$49\quad;\quad 11\quad;\quad 111\quad;\quad 1005692$$
    % \quad;\quad 99\quad;\quad59\quad;\quad 442
    \item Déterminer tous les diviseurs de $63$ et $84$ et déduire $63\wedge 84$. (1.5 pt)
    \item Déterminer $63\vee 84$. (1.5 pt)
\end{enumerate}

\underline{\large\textbf{Exercice 2 :}}(5 pt)

Soient $a$ et $b$ deux entiers naturels tel que : $a = 1176$ et $b = 2646$.
\begin{enumerate}
    \item Décomposer $a$ et $b$ en produit des facteurs premiers. (2 pt)
    \item En déduire $a\wedge b$ et $a\vee b$. (1 pt)
    \item Simplifier $\displaystyle\frac{b}{a}$ et $\sqrt{a\times b}$. (2 pt)
    
\end{enumerate}

\underline{\large\textbf{Exercice 3 :}}(1.5 pt)

 Montrer que les opérations suivantes sont fausses (Sans utiliser la calculatrice.) :
    \begin{enumerate}
        \item $4593 \times 15937 = 73198642$
        \item $426^{9} = 14398744591$
        \item $2456321 + 458965123 = 156978484153$
    \end{enumerate}


\underline{\large\textbf{Exercice 4 :}}(6 pt)

Soit $ABCD$ un parallélogramme. 
\begin{enumerate}
	\item Construire les points $E$ et $F$ tels que $\overrightarrow{AE} = \dfrac{3}{2}\overrightarrow{AB}$ et $\overrightarrow{AF} = 3\overrightarrow{AD}$. (1pt)
	\item Complétez $\overrightarrow{AB} = \dots$ et $\overrightarrow{AD} = \dots$.(0.5 pt)
	\item Montrer que $\overrightarrow{AD} = \overrightarrow{AC} - \overrightarrow{AB}$. (0.5 pt)
	\item Montrer que $\overrightarrow{CE} = \dfrac{3}{2}\overrightarrow{AB} - \overrightarrow{AC}$ et $\overrightarrow{FE} = \dfrac{9}{2}\overrightarrow{AB} - 3\overrightarrow{AC}$.(1pt + 1.5pt)
	\item Montrer que $\overrightarrow{FE} = 3\overrightarrow{CE}$. (0.75 pt)
	\item En déduire que les points $C,\ E$ et $F$ sont alignés. (0.75 pt)
\end{enumerate}

\underline{\large\textbf{Exercice 5 :}}(1.5 pt)

Soit $\overrightarrow{u}$ un vecteur du plan. Construire les vecteurs : $\dfrac{4}{5}\overrightarrow{u}$, $-\overrightarrow{u}$ et $-\dfrac{2}{3}\overrightarrow{u}$.

%%%$E,$ $J$ et $P$ les points tels que : $\overrightarrow{AE} = \displaystyle\frac{3}{2}\overrightarrow{AB}\ \ ;\ \ \overrightarrow{AF} = 3\overrightarrow{AC}\ \ ;\ \  \overrightarrow{AP} = 2\overrightarrow{AB} - \overrightarrow{AC}$.
%%%\begin{enumerate}
%%%    \item Montrer que $\overrightarrow{CP} = 2\overrightarrow{CB}$. (0.75 pt)
%%%    \item Construire une figure convenable. (1 pt)
%%%    \item Montrer que $\overrightarrow{EF} = -\displaystyle\frac{3}{2}\overrightarrow{AB} + 3\overrightarrow{AC}$ et $\overrightarrow{PF} = -2\overrightarrow{AB} + 4\overrightarrow{AC}$. (1 pt)
%%%    \item En déduire que  $2\overrightarrow{EF} = \displaystyle\frac{3}{2}\overrightarrow{PF}$. (0.25 pt)
%%%    \item En déduire que les points $E, F$ et $P$ sont alignés. (0.5 pt)
%%%    \item Soit $I$ le milieu du segment $[AB]$, montrer que $\overrightarrow{EF}$ et $\overrightarrow{CI}$ sont colinéares. (1.5 pt)
%%\end{enumerate}

%\underline{\large\textbf{Exercice 5 :}}(2 pt)
%
%Soit $ABC$ un triangle, et $M$ et $N$ deux points du plan tel que : $\overrightarrow{AM} = \displaystyle\frac{2}{3}\overrightarrow{AC}$ et $\overrightarrow{AN} = \displaystyle\frac{3}{2}\overrightarrow{AB}$.
%
%Soit $M^{'}$ le projeté de $M$ sur $(BC)$ parraléllemnt à $(AB)$ et  $N^{'}$ le projeté de $N$ sur $(AC)$ parraléllemnt à $(BC)$ 
%
%parallèlement à $(BC)$.
%\begin{enumerate}
%    \item Construire les points $M,N,M^{'}$ et $N^{'}$.
%\end{enumerate}
%
%
%
%\underline{\large\textbf{Exercice 6 :}}(2 pt)
%\begin{enumerate}
%	\item Soient $a$ et $b$ deux entiers naturels. Complétez $a^2 - b^2 = \dots\dots\dots$ (0.5pt)
%    \item En déduire que $A = 3^{20} - 2^{30}$ n'est pas premier. (1.5 pt)
%\end{enumerate}



\end{document}