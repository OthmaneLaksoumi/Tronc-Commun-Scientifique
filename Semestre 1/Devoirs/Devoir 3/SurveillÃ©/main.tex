\documentclass[12pt,a4paper]{article}
\usepackage[top=2cm,left=1cm,right=1cm,bottom=2cm]{geometry}
\usepackage{tikz}
 \usepackage[tikz]{bclogo}%
 \usetikzlibrary{calc,intersections}
\usetikzlibrary{arrows.meta}
\usetikzlibrary{decorations.pathmorphing}
\usepackage{amssymb,mathtools,amsthm}
\usepackage{fourier}
\usepackage{tkz-tab}
\usepackage{xcolor}
\usepackage{multicol, array, fancyhdr}
\usepackage{tasks}
\newcommand{\Lim}{\displaystyle\lim}

\begin{document}

\pagestyle{fancy}
\fancyhf{} % clear all header and footer fields
\fancyhead[L]{Lycée : Zitoun \hspace{1.5cm} Année scolaire : 2024-2025} % Left header
\fancyhead[C]{ \hspace{4cm} Niveau : TCSF} % Right-Center header
\fancyhead[R]{Prof. Othmane Laksoumi} % Right header
%\fancyfoot[C]{\thepage} % Footer


\begin{center}
    \textbf{\Large  Devoir Surveillé 3 Version A}
\end{center}

\underline{\large\textbf{Exercice 1 :}}(6 pt)
Soient $A\left(\dfrac{2}{5};1\right)$ ; $B\left(1;\dfrac{1}{2}\right)$ ; $C(-\sqrt{2}; 2)$.
\begin{enumerate}
	\item Déterminer les coordonnés des vecteurs $\overrightarrow{AB}$, $\overrightarrow{AC}$ et $\overrightarrow{BC}$. (1 pt + 1 pt + 1 pt)
	\item Calculer $det(\overrightarrow{AB}; \overrightarrow{AC})$, $det(\overrightarrow{AB}; \overrightarrow{BC})$ (1.5 pt + 1.5 pt)
\end{enumerate}

\underline{\large\textbf{Exercice 2 :}}(7 pt)
\begin{enumerate}
	\item Étudier la colinéarité des vecteurs suivants :
		\begin{enumerate}
			\item $\vec{u}(\sqrt{3};-2)$ ; $\vec{v}(-\sqrt{6}; \sqrt{8})$. (1 pt)
			\item $\vec{u}\left(\dfrac{6}{5};\dfrac{-7}{2}\right)$ ; $\vec{v}\left(1; \dfrac{5}{9}\right)$. (1 pt)
		\end{enumerate}
	\item Étudier l'alignement des points suivants :
		\begin{enumerate}
			\item $A\left(1;\dfrac{4}{5}\right)$ ; $B(-1;1)$ ; $C\left(-9;\dfrac{9}{5}\right)$. (1.5 pt)
			\item $A(0;-\sqrt{3})$ ; $B(\sqrt{2};0)$ ; $C\left(\sqrt{3};\sqrt{2} - \sqrt{3}\right)$. (1.5 pt)
		\end{enumerate}
	\item Construire la droite d'équation : $x = 2$. (1 pt)
	\item Construire la droite d'équation : $y = -\dfrac{3}{2}$. (1 pt)
\end{enumerate}

\underline{\large\textbf{Exercice 3 :}}(3 pt)
\begin{enumerate}
	\item Déterminer une représentation paramétrique de la droite passant par le point $A\left(\sqrt{3}; 1\right)$ et dirigée par le vecteur $\vec{u}\left(\dfrac{1}{2};\dfrac{3}{2}\right)$. (1 pt)
	\item Déterminer une représentation paramétrique de la droite $(AB)$, avec $A(0;1)$ et $B(-2;3)$. (2 pt)
\end{enumerate}

\underline{\large\textbf{Exercice 4 :}}(3 pt)
\begin{enumerate}
	\item Déterminer une équation cartésienne de la droite $(D)$ passant par le point $A(-1;1)$ et dirigée par le vecteur $\vec{u}(1;2)$. (2 pt)
	\item Déterminer un point appartenant à la droite $(D)$. (1 pt)
	\item Le point $B\left(\dfrac{5}{6}; \dfrac{14}{3}\right)$ appartenant à la droite $(D)$? (0.5 pt)
	\item Le point $C\left(\dfrac{1}{2}; 3\right)$ appartenant à la droite $(D)$? (0.5 pt)
\end{enumerate}

\newpage

\begin{center}
    \textbf{\Large  Devoir Surveillé 3 Version B}
\end{center}

\underline{\large\textbf{Exercice 1 :}}(6 pt)
Soient $A\left(-\dfrac{3}{4};1\right)$ ; $B\left(1;\dfrac{1}{2}\right)$ ; $C(\sqrt{2}; 1)$.
\begin{enumerate}
	\item Déterminer les coordonnés des vecteurs $\overrightarrow{AB}$, $\overrightarrow{AC}$ et $\overrightarrow{BC}$. (1 pt + 1 pt + 1 pt)
	\item Calculer $det(\overrightarrow{AB}; \overrightarrow{AC})$, $det(\overrightarrow{AB}; \overrightarrow{BC})$ (1.5 pt + 1.5 pt)
\end{enumerate}

\underline{\large\textbf{Exercice 2 :}}(7 pt)
\begin{enumerate}
	\item Étudier la colinéarité des vecteurs suivants :
		\begin{enumerate}
			\item $\vec{u}(-\sqrt{6};\sqrt{8})$ ; $\vec{v}(\sqrt{3}; -2)$. (1 pt)
			\item $\vec{u}\left(1;\dfrac{5}{9}\right)$ ; $\vec{v}\left(\dfrac{6}{5}; \dfrac{-7 }{2}\right)$. (1 pt)
		\end{enumerate}
	\item Étudier l'alignement des points suivants :
		\begin{enumerate}
			\item $A\left(2;\dfrac{1}{2}\right)$ ; $B\left(\dfrac{1}{2};1\right)$ ; $C\left(3;-1\right)$. (1.5 pt)
			\item $A(\sqrt{2};0)$ ; $B(0;-\sqrt{3})$ ; $C\left(\sqrt{3};\sqrt{2} - \sqrt{3}\right)$. (1.5 pt)
		\end{enumerate}
	\item Construire la droite d'équation : $x = \dfrac{3}{2}$. (1 pt)
	\item Construire la droite d'équation : $y = -2$. (1 pt)
\end{enumerate}

\underline{\large\textbf{Exercice 3 :}}(3 pt)
\begin{enumerate}
	\item Déterminer une représentation paramétrique de la droite passant par le point $A\left(-\sqrt{5}; 2\right)$ et dirigée par le vecteur $\vec{u}\left(\dfrac{5}{4};\dfrac{9}{8}\right)$. (1 pt)
	\item Déterminer une représentation paramétrique de la droite $(AB)$, avec $A(-2;3)$ et $B(0;1)$. (2 pt)
\end{enumerate}

\underline{\large\textbf{Exercice 4 :}}(3 pt)
\begin{enumerate}
	\item Déterminer une équation cartésienne de la droite $(D)$ passant par le point $A(2;-2)$ et dirigée par le vecteur $\vec{u}(3;4)$. (2 pt)
	\item Déterminer un point appartenant à la droite $(D)$. (1 pt)
	\item Le point $B\left(\dfrac{5}{4}; -3\right)$ appartenant à la droite $(D)$? (0.5 pt)
	\item Le point $C\left(\dfrac{1}{4}; 3\right)$ appartenant à la droite $(D)$? (0.5 pt)
\end{enumerate}

















\end{document}