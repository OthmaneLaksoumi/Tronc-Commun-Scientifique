\documentclass[12pt,a4paper]{article}
\usepackage[top=2cm,left=1cm,right=1cm,bottom=2cm]{geometry}
\usepackage{tikz}
 \usepackage[tikz]{bclogo}%
 \usetikzlibrary{calc,intersections}
\usetikzlibrary{arrows.meta}
\usetikzlibrary{decorations.pathmorphing}
\usepackage{amssymb,mathtools,amsthm}
\usepackage{fourier}
\usepackage{tkz-tab}
\usepackage{xcolor}
\usepackage{multicol, array, fancyhdr}
\usepackage{tasks}
\newcommand{\Lim}{\displaystyle\lim}

\begin{document}


\pagestyle{fancy}
\fancyhf{} % clear all header and footer fields
\fancyhead[L]{Lycée : Zitoun \hspace{1.5cm} Année scolaire : 2024-2025} % Left header
\fancyhead[C]{ \hspace{4cm} Niveau : TCSF} % Right-Center header
\fancyhead[R]{Prof. Othmane Laksoumi} % Right header
\fancyfoot[C]{\thepage} % Footer


\begin{center}
    \textbf{\Large  Devoir Libre 3}
\end{center}

\underline{\large\textbf{Exercice 1 :}}(7 pt)
Soient $A(1;-2)$ ; $B(-3;5)$ ; $C\left(\sqrt{2}; \dfrac{1}{2}\right)$ ; $D(0;-\dfrac{2}{3})$.
\begin{enumerate}
	\item Déterminer les coordonnés des vecteurs $\overrightarrow{AB}$, $\overrightarrow{AC}$, $\overrightarrow{BC}$ et $\overrightarrow{BD}$. (1 pt + 1 pt + 1 pt + 1 pt)
	\item Calculer $det(\overrightarrow{AB}; \overrightarrow{AC})$, $det(\overrightarrow{BC}; \overrightarrow{BD})$ (1.5 pt + 1.5 pt)
\end{enumerate}

\underline{\large\textbf{Exercice 2 :}}(7 pt)
\begin{enumerate}
	\item Étudier la colinéarité des vecteurs suivants :
		\begin{enumerate}
			\item $\vec{u}(\sqrt{2};-1)$ ; $\vec{v}(-\sqrt{6}; \sqrt{3})$. (1 pt)
			\item $\vec{u}\left(\dfrac{5}{6};\dfrac{-3}{2}\right)$ ; $\vec{v}\left(-5; \dfrac{4}{7}\right)$. (1 pt)
		\end{enumerate}
	\item Étudier l'alignement des points suivants :
		\begin{enumerate}
			\item $A\left(\dfrac{1}{2};\dfrac{-5}{3}\right)$ ; $B(1;-1)$ ; $C\left(\dfrac{5}{3};-5\right)$. (1.5 pt)
			\item $A(1;0)$ ; $B(0;-1)$ ; $C\left(\dfrac{-5}{2};\dfrac{-7}{2}\right)$. (1.5 pt)
		\end{enumerate}
	\item Construire les droites des équations : $x = \dfrac{3}{2}$ et $y = -\dfrac{1}{2}$. (1 pt + 1 pt)
\end{enumerate}

\underline{\large\textbf{Exercice 3 :}}(3 pt)
\begin{enumerate}
	\item Déterminer une représentation paramétrique de la droite passant par le point $A\left(\dfrac{1}{2}; 1\right)$ et dirigée par le vecteur $\vec{u}\left(1;\dfrac{3}{2}\right)$. (1 pt)
	\item Déterminer une représentation paramétrique de la droite $(AB)$, avec $A(1;2)$ et $B(-1;6)$. (2 pt)
\end{enumerate}

\underline{\large\textbf{Exercice 4 :}}(3 pt)
\begin{enumerate}
	\item Déterminer une équation cartésienne de la droite $(D)$ passant par le point $A(1;2)$ et dirigée par le vecteur $\vec{u}(-1;-3)$. (2 pt)
	\item Le point $B\left(\dfrac{2}{3}; 1\right)$ appartenant à la droite $(D)$. (0.5 pt)
	\item Le point $B\left(\dfrac{11}{3}; \dfrac{-5}{3}\right)$ appartenant à la droite $(D)$. (0.5 pt)
\end{enumerate}


















\end{document}