\documentclass[12pt,a4paper]{article}
\usepackage[top=2cm,left=1cm,right=1cm,bottom=2cm]{geometry}
\usepackage{tikz}
 \usepackage[tikz]{bclogo}%
 \usetikzlibrary{calc,intersections}
\usetikzlibrary{arrows.meta}
\usetikzlibrary{decorations.pathmorphing}
\usepackage{amssymb,mathtools,amsthm}
\usepackage{fourier}
\usepackage{tkz-tab}
\usepackage{xcolor}
\usepackage{multicol, array, fancyhdr}
\usepackage{tasks}
\newcommand{\Lim}{\displaystyle\lim}

\begin{document}


\pagestyle{fancy}
\fancyhf{} % clear all header and footer fields
\fancyhead[L]{Lycée : Zitoun \hspace{1.5cm} Année scolaire : 2024-2025} % Left header
\fancyhead[C]{ \hspace{4cm} Niveau : TCSF} % Right-Center header
\fancyhead[R]{Prof. Othmane Laksoumi} % Right header
\fancyfoot[C]{\thepage} % Footer


\begin{center}
    \textbf{\Large  Devoir Libre 2}
\end{center}

\underline{\large\textbf{Exercice 1 :}}(9.5 pt)
\begin{enumerate}
   \item Compléter avec l'un des symboles $\in$ ou $\not\in$ : $\dfrac{24}{3}\dots \mathbb{N}$ \ \ ; \ \ $\dfrac{-2}{6}\dots \mathbb{Q}$ \ \ ; \ \ $\dfrac{18}{8}\dots \mathbb{Z}$ \ \ ; \ \ $\dfrac{4}{10^2}\dots \mathbb{D}$. (1 pt)
   \item Calculer les expressions suivantes et donner le résultat sous forme de fraction irréductible : (1 pt + 1 pt + 1 pt + 1 pt)
   $$\hspace{-10mm} A = \dfrac{4}{3} - \dfrac{11}{6} - \dfrac{1}{3}\left(1 - \dfrac{4}{3}\right) \ \ ; \ \ B = \dfrac{1}{7} + \dfrac{11}{3}\times\dfrac{2}{3} - \dfrac{1}{2}\left(3 - \dfrac{5}{9} \right) \ \ ; \ \ C = \dfrac{2 - \dfrac{2}{3} + \dfrac{3}{2}}{2 + \dfrac{2}{3} - \dfrac{3}{2}}\ \ ; \ \ D = \dfrac{(2^3\times 7^5)^{-2}}{4^3\times 77^{-10}}\times\dfrac{11^{-10}}{3^2}$$
   \item Calculer le nombre réel x sachant que : ( 1.5 pt)
   $$1 + \dfrac{2}{3} + \dfrac{3}{4} + \dfrac{4}{5} + \dfrac{1}{x} = 2$$
   \item Donner l’écriture scientifique des nombres suivants : (0.5 pt + 0.5 pt)
   $$A = 27\times 10^{-4}\times 0,0015 \quad ; \quad B = \dfrac{810\times 10^{-3} \times (3\times 10)^{-2}}{0,0003}$$
   \item Développer et simplifier : (1 pt + 1 pt)
   $$A = 2(x + 1)^3 - (1 - 2x)^3 - (3x - 2)^2$$
   \item $$B = 27x^3 + 64 + (2x + 5)(3x + 4)$$
   
\end{enumerate}

\underline{\large\textbf{Exercice  2:}}(3.5 pt)
\begin{enumerate}
	\item Résoudre dans $\mathbb{R}$ : (0.5 pt + 0.5 pt + 0.5 pt  + 0.5 pt )
		$$(E_1) \ : \ |4x - 5| = 3 \quad ; \quad (E_2) \ : \ |x - 1| = -1 \quad;\quad (I_1) : |2x - 6| \leq 4 \quad;\quad |3x - 8| \geq 1$$
	\item Écrire sous forme d’un intervalle les ensembles suivants : (1,5 pt)
	$$x\leq 4 \quad;\quad -5\leq x\leq 2 \quad;\quad x \geq -5 \quad;\quad [-2;3]\cap \left[-1;\dfrac{9}{2}\right]\quad;\quad [-3;+\infty[\cap]-\infty;0[\quad;\quad \left[\dfrac{-5}{3};+\infty\right[\cap [0;+\infty[$$
\end{enumerate}

\underline{\large\textbf{Exercice  3:}}(3.5 pt)

Soit $x$ et $y$ deux réels tels que $2\leq x\leq 3$ et $-3\leq y\leq 1$.

Donner un encadrement de : $x + y\quad;\quad 5x - 3y \quad;\quad xy \quad;\quad x^2 - y^2 - 2x + 1$ (0.25 pt + 0.75 pt + 1 pt + 1.5 pt)

\underline{\large\textbf{Exercice  4:}}(3.5 pt)
\begin{enumerate}
	\item Soit $a,b\in\mathbb{R}^*_+$ tel que $\sqrt{\dfrac{a}{b}} + \sqrt{\dfrac{b}{a}} = \sqrt{5}$
   	\begin{enumerate}
   		\item Montrer que : $\dfrac{a}{b} + \dfrac{b}{a} = 3$. (1pt)
   		\item Calculer $\dfrac{a^2}{b^2} + \dfrac{b^2}{a^2}$ et $\dfrac{a^3}{b^3} + \dfrac{b^3}{a^3}$. (1pt + 1.5pt)
   	\end{enumerate}
\end{enumerate}


























\end{document}