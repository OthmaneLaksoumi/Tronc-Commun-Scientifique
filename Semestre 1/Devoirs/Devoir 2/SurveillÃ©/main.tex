\documentclass[12pt,a4paper]{article}
\usepackage[top=2cm,left=1cm,right=1cm,bottom=2cm]{geometry}
\usepackage{tikz}
 \usepackage[tikz]{bclogo}%
 \usetikzlibrary{calc,intersections}
\usetikzlibrary{arrows.meta}
\usetikzlibrary{decorations.pathmorphing}
\usepackage{amssymb,mathtools,amsthm}
\usepackage{fourier}
\usepackage{tkz-tab}
\usepackage{xcolor}
\usepackage{multicol, array, fancyhdr}
\usepackage{tasks}
\newcommand{\Lim}{\displaystyle\lim}

\begin{document}


\pagestyle{fancy}
\fancyhf{} % clear all header and footer fields
\fancyhead[L]{Lycée : Zitoun \hspace{1.5cm} Année scolaire : 2024-2025} % Left header
\fancyhead[C]{ \hspace{4cm} Niveau : TCSF} % Right-Center header
\fancyhead[R]{Prof. Othmane Laksoumi} % Right header
\fancyfoot[C]{\thepage} % Footer


\begin{center}
    \textbf{\Large  Devoir Surveillé 2 Version A}
\end{center}

\underline{\large\textbf{Exercice 1 :}}(9 pt)
\begin{enumerate}
   \item Compléter avec l'un des symboles $\in$ ou $\not\in$ : $\dfrac{9}{3}\dots \mathbb{N}$ \ \ ; \ \ $\dfrac{20}{9}\dots \mathbb{Q}$ \ \ ; \ \ $\dfrac{5}{2}\dots \mathbb{Z}$ \ \ ; \ \ $\dfrac{51}{10^{10}}\dots \mathbb{D}$. (1 pt)
   \item Calculer les expressions suivantes et donner le résultat sous forme de fraction irréductible : (1 pt + 1.5 pt)
   $$\hspace{-10mm} A = \dfrac{4}{3} - \dfrac{11}{6} - \dfrac{1}{2}\left(1 - \dfrac{4}{3}\right)  \ \ ; \ \ B = \dfrac{1 - \dfrac{4}{5} + \dfrac{5}{6}}{1 + \dfrac{4}{5} - \dfrac{5}{6}}$$
   \item Calculer le nombre réel x sachant que : ( 2 pt)
   $$-1 + \dfrac{1}{2} - \dfrac{2}{3} + \dfrac{3}{5} - \dfrac{2}{x} = 0$$
   \item Donner l’écriture scientifique des nombres suivants : (1 pt)
   $$A = 1203\times 10^{-3}\times 0,0001$$
   \item Développer et simplifier : (1 pt)
   $$A = (x + 2)^3 - (1 - 2x)^3 - \left(\dfrac{1}{2}x - 2\right)^2$$
   \item 
   		\begin{enumerate}
   			\item Développer : $B = (2x + 1)(4x^2 - 2x + 1)$ (0.5 pt)
   			\item Déduire une factorisation de l'expression $C = 8x^3 + 1 + (2x + 1)(-2x^2 + 1)$ (1 pt)
	    \end{enumerate}
   
\end{enumerate}

\underline{\large\textbf{Exercice  2:}}(3 pt)
\begin{enumerate}
	\item Résoudre dans $\mathbb{R}$ : (0.25 pt + 0.25 pt + 0.75 pt  + 0.75 pt )
		$$(E_1) \ : \ |-2x + 1| = 3 \quad ; \quad (E_2) \ : \ |x + 1| = -1 \quad;\quad (I_1) : |3x - 1| \leq 2 \quad;\quad (I_2) : |3x - 1| \geq 2$$
	\item Écrire sous forme d’un intervalle les ensembles suivants : (0.25 pt + 0.25 pt + 0.25 pt + 0.25 pt)
	$$x\leq \dfrac{1}{2} \quad;\quad -2\leq x\leq \sqrt{2} \quad;\quad [-3;+\infty[\cap]-\infty;0[\quad;\quad \left[\dfrac{-5}{3};+\infty\right[\cap [0;+\infty[$$
\end{enumerate}

\underline{\large\textbf{Exercice  3:}}(4 pt)

Soit $x$ et $y$ deux réels tels que $1\leq x\leq \dfrac{3}{2}$ et $-1\leq y\leq 2$.

Donner un encadrement de : $x + y\quad;\quad 2x - y \quad;\quad xy \quad;\quad x^2 + y^2 - \dfrac{5}{2}$ (0.25 pt + 0.75 pt + 1 pt + 2 pt)

\underline{\large\textbf{Exercice  4:}}(4 pt)
\begin{enumerate}
	\item Soit $a,b\in\mathbb{R}^*_+$ tel que $\sqrt{\dfrac{a}{b}} - \sqrt{\dfrac{b}{a}} = \sqrt{2}$
   	\begin{enumerate}
   		\item Montrer que : $\dfrac{a}{b} + \dfrac{b}{a} = 4$. (1pt)
   		\item Calculer $\dfrac{a^2}{b^2} + \dfrac{b^2}{a^2}$ et $\dfrac{a^3}{b^3} + \dfrac{b^3}{a^3}$. (1pt + 2pt)
   	\end{enumerate}
\end{enumerate}




\begin{center}
    \textbf{\Large  Devoir Surveillé 2 Version B}
\end{center}

\underline{\large\textbf{Exercice 1 :}}(9 pt)
\begin{enumerate}
   \item Compléter avec l'un des symboles $\in$ ou $\not\in$ : $\dfrac{8}{4}\dots \mathbb{N}$ \ \ ; \ \ $\dfrac{15}{4}\dots \mathbb{Q}$ \ \ ; \ \ $\dfrac{50}{2}\dots \mathbb{Z}$ \ \ ; \ \ $\dfrac{5}{10^{2}}\dots \mathbb{D}$. (1 pt)
   \item Calculer les expressions suivantes et donner le résultat sous forme de fraction irréductible : (1 pt + 1.5 pt)
   $$\hspace{-10mm} A = \dfrac{4}{3} - \dfrac{11}{6} - \dfrac{1}{2}\left(1 - \dfrac{4}{3}\right)  \ \ ; \ \ B = \dfrac{1 - \dfrac{5}{6} + \dfrac{6}{7}}{1 + \dfrac{5}{6} - \dfrac{6}{7}}$$
   \item Calculer le nombre réel x sachant que : ( 2 pt)
   $$1 - \dfrac{1}{2} + \dfrac{2}{3} - \dfrac{3}{5} + \dfrac{2}{x} = 0$$
   \item Donner l'écriture scientifique des nombres suivants : (1 pt)
   $$A = 2001\times 10^{-3}\times 0,0001$$
   \item Développer et simplifier : (1 pt)
   $$A = (x - 2)^3 - (1 + 2x)^3 - \left(\dfrac{1}{2}x - 2\right)^2$$
   \item 
   		\begin{enumerate}
   			\item Développer : $B = (2x - 1)(4x^2 + 2x + 1)$ (0.5 pt)
   			\item Déduire une factorisation de l'expression $C = 8x^3 - 1 + (2x - 1)(-2x^2 + 1)$ (1 pt)
	    \end{enumerate}
   
\end{enumerate}

\underline{\large\textbf{Exercice  2:}}(3 pt)
\begin{enumerate}
	\item Résoudre dans $\mathbb{R}$ : (0.25 pt + 0.25 pt + 0.75 pt  + 0.75 pt )
		$$(E_1) \ : \ |-4x + 1| = 3 \quad ; \quad (E_2) \ : \ |x + 1| = -1 \quad;\quad (I_1) : |7x - 8| \leq 6 \quad;\quad (I_2) : |7x - 8| \geq 6$$
	\item Écrire sous forme d’un intervalle les ensembles suivants : (0.25 pt + 0.25 pt + 0.25 pt + 0.25 pt)
	$$x\geq \dfrac{2}{5} \quad;\quad 0\leq x\leq \sqrt{5} \quad;\quad [-3;+\infty[\cap]-\infty;0[\quad;\quad \left[\dfrac{-5}{3};+\infty\right[\cap [0;+\infty[$$
\end{enumerate}

\underline{\large\textbf{Exercice  3:}}(4 pt)

Soit $x$ et $y$ deux réels tels que $\dfrac{1}{2}\leq x\leq 3$ et $-2\leq y\leq 1$.

Donner un encadrement de : $x + y\quad;\quad 2x - y \quad;\quad xy \quad;\quad x^2 + y^2 - \dfrac{5}{2}$ (0.25 pt + 0.75 pt + 1 pt + 2 pt)

\underline{\large\textbf{Exercice  4:}}(4 pt)
\begin{enumerate}
	\item Soit $a,b\in\mathbb{R}^*_+$ tel que $\sqrt{\dfrac{a}{b}} - \sqrt{\dfrac{b}{a}} = \sqrt{3}$
   	\begin{enumerate}
   		\item Montrer que : $\dfrac{a}{b} + \dfrac{b}{a} = 5$. (1pt)
   		\item Calculer $\dfrac{a^2}{b^2} + \dfrac{b^2}{a^2}$ et $\dfrac{a^3}{b^3} + \dfrac{b^3}{a^3}$. (1pt + 2pt)
   	\end{enumerate}
\end{enumerate}



























\end{document}