\documentclass[12pt,a4paper]{article}
\usepackage[top=2cm,left=1cm,right=1cm,bottom=2cm]{geometry}
\usepackage{tikz}
 \usepackage[tikz]{bclogo}%
 \usetikzlibrary{calc,intersections}
\usetikzlibrary{arrows.meta}
\usetikzlibrary{decorations.pathmorphing}
\usepackage{amssymb,mathtools,amsthm}
\usepackage{fourier}
\usepackage{tkz-tab}
\usepackage{xcolor}
\usepackage{multicol, array, fancyhdr}
\usepackage{tasks}
\newcommand{\Lim}{\displaystyle\lim}

\begin{document}


\pagestyle{fancy}
\fancyhf{} % clear all header and footer fields
\fancyhead[L]{Lycée : Zitoun \hspace{1.5cm} Année scolaire : 2024-2025} % Left header
\fancyhead[C]{ \hspace{4cm} Niveau : TCSF} % Right-Center header
\fancyhead[R]{Prof. Othmane Laksoumi} % Right header
\fancyfoot[C]{\thepage} % Footer


\begin{center}
    \textbf{\Large  Devoir Libre 1}
\end{center}

\underline{\large\textbf{Exercice 1 :}}(4.5 pt)
\begin{enumerate}
    % \item Déterminer les nombres pairs et les nombres impairs parmi les nombres suivants : (0.25 + 0.25 + 0.25 + 0.25 pt)
    \item Soit $n$ un entier naturel. Étudier la parité des nombres suivants : ( 1.5 pt)
    $$A = 16n + 4\quad;\quad B = 4n^2 + 8n + 1 \quad;\quad C = (4n + 9)(6n + 11).$$
    % $$536\times 123\quad;\quad 33\times 1399\quad;\quad 15559 + 125699\quad;\quad 33369 - 4890$$
    \item Déterminer les nombres premiers parmi les nombres suivants :(1 pt)
    $$543\quad;\quad 111\quad;\quad 2\quad;\quad 10095$$
    % \quad;\quad 99\quad;\quad59\quad;\quad 442
    \item Déterminer tous les diviseurs de $210$ et $315$ et déduire $210\wedge 315$. (1 pt)
    \item Déterminer $35\vee 210$. (1 pt)
\end{enumerate}

\underline{\large\textbf{Exercice 2 :}}(4 pt)

Soient $a$ et $b$ deux entiers naturels tel que : $a = 1200$ et $b = 5292$.
\begin{enumerate}
    \item Décomposer $a$ et $b$ en produit des facteurs premiers. (2 pt)
    \item En déduire $a\wedge b$ et $a\vee b$. (1 pt)
    \item Simplifier $\displaystyle\frac{a}{b}$ et $\sqrt{a\times b}$. (2 pt)
    
\end{enumerate}


\underline{\large\textbf{Exercice 3 :}}(5 pt)

Soit $ABC$ un triangle, $E,$ $J$ et $P$ les points tels que : $\overrightarrow{AE} = \displaystyle\frac{3}{2}\overrightarrow{AB}\ \ ;\ \ \overrightarrow{AF} = 3\overrightarrow{AC}\ \ ;\ \  \overrightarrow{AP} = 2\overrightarrow{AB} - \overrightarrow{AC}$.
\begin{enumerate}
    \item Montrer que $\overrightarrow{CP} = 2\overrightarrow{CB}$. (0.75 pt)
    \item Construire une figure convenable. (1 pt)
    \item Montrer que $\overrightarrow{EF} = -\displaystyle\frac{3}{2}\overrightarrow{AB} + 3\overrightarrow{AC}$ et $\overrightarrow{PF} = -2\overrightarrow{AB} + 4\overrightarrow{AC}$. (1 pt)
    \item En déduire que  $2\overrightarrow{EF} = \displaystyle\frac{3}{2}\overrightarrow{PF}$. (0.25 pt)
    \item En déduire que les points $E, F$ et $P$ sont alignés. (0.5 pt)
    \item Soit $I$ le milieu du segment $[AB]$, montrer que $\overrightarrow{EF}$ et $\overrightarrow{CI}$ sont colinéares. (1.5 pt)
\end{enumerate}

\underline{\large\textbf{Exercice 4 :}}(3 pt)

Soit $ABC$ un triangle, et $M$ un point du plan tel que : $\overrightarrow{AM} = \displaystyle\frac{4}{5}\overrightarrow{AC}$ et $M^{'}$ le projeté de $M$ sur $(AB)$ 

parallèlement à $(BC)$.
\begin{enumerate}
    \item Construire une figure convenable. (0.75 pt)
    \item Montrer que : $\overrightarrow{AM^{'}} = \displaystyle\frac{4}{5}\overrightarrow{AB}$ (1.25 pt)
    \item Montrer par le théorème de Thalés réciproque que : $(MM^{'}) // (BC)$. (1 pt)
\end{enumerate}



\underline{\large\textbf{Exercice 5 :}}(3.5 pt)
\begin{enumerate}
    \item Montrer que $A = 2^{120} - 1$ n'est pas premier. (1.5 pt)
    \item Montrer que $B = (99^{99} - 1)^2 + 2\times 99^{99} - 1$ n'est pas premier. (2 pt)
\end{enumerate}



\end{document}