\documentclass[12pt,a4paper]{article}
\usepackage[top=2cm,left=1cm,right=1cm,bottom=2cm]{geometry}
\usepackage{amssymb,mathtools,amsthm}
\usepackage{fourier}
\usepackage{xcolor}
\usepackage{multicol, array, fancyhdr}
\usepackage{tasks}
\newcommand{\Lim}{\displaystyle\lim}
\renewcommand{\columnseprule}{1pt}
\renewcommand{\arraystretch}{1.5}
% \renewcommand{\frac}[2]{\displaystyle\frac{#1}{#2}}


%======================================================
\newtheoremstyle{mystyle}
  {\topsep}% espace avant
  {\topsep}% espace après
  {\upshape}% police du corps du théorème
  {}% indentation (vide pour rien, \parindent)
  {\bfseries\sffamily}% police du titre du théorème
  { :}% ponctuation après le théorème
  { }% après le titre du théorème (espace ou \newline)
  {%
    \rule[0.5\baselineskip]{0.5\textwidth}{1pt}%
    \newline\fcolorbox{black}{white}{%
      \thmname{#1}\thmnumber{ \textup{#2}}\thmnote{ \textnormal{(#3)}}%
    }%
    
    % \vspace{0.5em} % Adjust vertical space after the title
  }% spécifications du titre

\theoremstyle{mystyle}
\newtheorem{exo}{Exercice}

%======================================================



\begin{document}


\pagestyle{fancy}
\fancyhf{} % clear all header and footer fields
\fancyhead[L]{Lycée : Zitoun \hspace{1.5cm} Année scolaire : 2024-2025} % Left header
\fancyhead[C]{ \hspace{4cm} Niveau : TCS} % Right-Center header
\fancyhead[R]{Prof. Othmane Laksoumi} % Right header
\fancyfoot[C]{\thepage} % Footer


\begin{center}
    \textbf{\Large Ensemble des nombres entiers naturels et notions d'arithmétique }
\end{center}
\begin{multicols*}{2}

\begin{exo}
    Déterminer les nombres pairs et les nombres impairs parmi les nombres suivants :\\
    \begin{tabular*}{0.5\textwidth}{@{\extracolsep{\fill}} ccc}
        $55^3$ & $15^4+21^7$ & $11^5-11^4$  \\
        $45591\times 1789054$ & $15^4+21^7$ & $753\times 457$
    \end{tabular*}
\end{exo}

\begin{exo}
    Soit $n$ un entier naturel.\\
    Déterminer les nombres pairs et les nombres impairs parmi les nombres suivants :\\
    \begin{tabular*}{0.5\textwidth}{@{\extracolsep{\fill}} ccc}
        $A = 8n + 6$ & $B = 6n + 7$ & $C = 4n + 2$  \\
        $D = 2n^2 + 4n + 1$ & $E = (2n+1)^2 - 2n$ & $F = 2n^2+4n+5$
    \end{tabular*}
\end{exo}



\begin{exo}
    \leavevmode
    % \vspace{-6mm}
    \begin{enumerate}
        \item Montrer que $165$ est un multiple de $33$.
        \item Montrer que $165$ est un multiple de $5$, $11$ et $55$.
        \item Montrer que $5\times 24$ est un diviseur de $120\times 28$.
        \item Montrer que $25^2\times 3$ est un diviseur de $15\times 750$.
    \end{enumerate}
\end{exo}

\begin{exo}
    Soit $n$ et $k$ deux entiers naturels.
    \begin{enumerate}
        \item Montrer que si $n=5k+1$ alors $n^2-1$ est divisible par $5$.
        \item Montrer que si $n=5k+2$ alors $n^2+1$ est divisible par $5$.
    \end{enumerate}
\end{exo}

\begin{exo}
    Sans calculer, les nombres suivants sont-ils premiers ?
    \begin{tasks}(2)
        \task $A = 55\times 49 + 5$
        \task $B = 35 \times 40 + 5^4$
        \task $C = 55^3\times 9 + 11^2 + 22$
    \end{tasks}
\end{exo}

\begin{exo}
    Calculer $a\wedge b$ et $a\vee b$ dans chacun des cas suivants :
    \begin{tasks}(2)
        \task[$\bullet$] $a = 7$ et $b=35$.
        \task[$\bullet$] $a = 13$ et $b = 17$.
        \task[$\bullet$] $a = 4$ et $b = 5$.
        \task[$\bullet$] $a = 36$ et $b = 15$.
    \end{tasks}
\end{exo}
\columnbreak 
\begin{exo}
 Montrer que les opérations suivantes sont fausses :
    \begin{enumerate}
        \item $156985 \times 4558912 = 45899665471$
        \item $156^{45} = 143987451874578591$
        \item $2456321 + 458965123 = 156978484153$
    \end{enumerate}
  
\end{exo}

\begin{exo}
Soit $n$ en entier naturel non nul.
   \begin{enumerate}
        \item Décomposer les nombres suivants en produit de facteurs premiers :
           \begin{tasks}(2)
               \task[$\bullet$] $A = 3^{n+2} + 3^n$
               \task[$\bullet$] $B = 5^{n+3} - 5^n$
           \end{tasks}
        \item En Déduire $A\wedge B$ et $A\vee B$.
   \end{enumerate}
   
\end{exo}

\begin{exo}

    Soit $m$ un entier naturel.
    \begin{enumerate}
        \item Montrer que si $m$ est pair, alors $m^2$ est pair.
        \item Montrer que si $m$ est impair, alors $m^2$ est impair.
    \end{enumerate}
\end{exo}

\begin{exo}
    \leavevmode
    % \vspace{-6mm}
    \begin{enumerate}
        \item Décomposer les deux nombres $a$ et $b$ en\\ produit de facteurs premiers, tels que : $a = 1386$ et $b = 2520$
        \item En déduire $a\wedge b$ et $a\vee b$.
    \end{enumerate}
\end{exo}

\begin{exo}
    \leavevmode
    % \vspace{-6mm}
    \begin{enumerate}
        \item Déterminer tous les nombres premiers\\inférieurs à $50$.
        \item Déterminer tous les nombres premiers $p$\\inférieurs à $50$ tel que $8$ divise $p-1$.
    \end{enumerate}
\end{exo}

\begin{exo}
   \leavevmode
    \begin{enumerate}
        \item Calculer $x\wedge y$ et $x\vee y$ dans chacun des cas\\ suivants :
        \vspace{-2.5mm}\begin{tasks}(2)
            \task[$\bullet$] $x=13$ et $y=15$ 
            \task[$\bullet$] $x=99$ et $y=33$ 
        \end{tasks}\vspace{-2.5mm}
        \item Comparer $(x\wedge y)\times(x\vee y)$ et $xy$ dans chaque cas précédent.
    \end{enumerate}
\end{exo}

\begin{exo}
    Montrer que $A = 40^2+40+41$ n'est pas un nombre premier.\\
    Montrer que $B = 2^{10} - 1$ n'est pas un nombre premier. $(10 = 5 \times 2)$
\end{exo}


\end{multicols*}



\end{document}
