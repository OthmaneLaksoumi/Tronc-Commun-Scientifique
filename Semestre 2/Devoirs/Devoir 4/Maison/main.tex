\documentclass[12pt,a4paper]{article}
\usepackage[top=2cm,left=1cm,right=1cm,bottom=2cm]{geometry}
\usepackage{tikz}
 \usepackage[tikz]{bclogo}%
 \usetikzlibrary{calc,intersections}
\usetikzlibrary{arrows.meta}
\usetikzlibrary{decorations.pathmorphing}
\usepackage{amssymb,mathtools,amsthm}
\usepackage{fourier}
\usepackage{tkz-tab}
\usepackage{xcolor}
\usepackage{multicol, array, fancyhdr}
\usepackage{tasks}
\newcommand{\Lim}{\displaystyle\lim}

\begin{document}


\pagestyle{fancy}
\fancyhf{} % clear all header and footer fields
\fancyhead[L]{Lycée : Zitoun \hspace{1.5cm} Année scolaire : 2024-2025} % Left header
\fancyhead[C]{ \hspace{4cm} Niveau : TCSF} % Right-Center header
\fancyhead[R]{Prof. Othmane Laksoumi} % Right header
\fancyfoot[C]{\thepage} % Footer


\begin{center}
    \textbf{\Large  Devoir Libre 4}
\end{center}

\underline{\large\textbf{Exercice 1 :}}(4 pt)
On considère le polynôme \[P(x) = x^3 - \dfrac{1}{2}x^2 -\dfrac{5}{2} x - 1\]

\begin{enumerate}
    \item Montrer que \(-1\) est une racine de \(P(x)\). (0.5 pt)
    \item Déterminer le polynôme \(Q(x)\) tel que \(P(x) = (x + 1)Q(x)\). (1 pt)
    \item Montrer que \(Q(x)\) est divisible par \((x - 2)\). (0.75 pt)
    \item Factoriser le polynôme \(Q(x)\). (1 pt)
    \item Donner une factorisation de \(P(x)\) en produit de trois binômes. (0.75 pt)
\end{enumerate}
\underline{\large\textbf{Exercice 2 :}}(6 pt)
\begin{enumerate}
	\item Résoudre dans $\mathbb{R}$ les équations : \\ $(E_1)\ : \ 3x^2 - x - 1 = 0$ \ ; \ $(E_2)\ : \ -x^2 - \dfrac{\sqrt{2}}{2}x - 1 = 0$ \ et \ $(E_3)\ : \ x^2 - \sqrt{3}x + \dfrac{3}{4} = 0$ (3 pt)
	\item 
		\begin{enumerate}
			\item Montrer que $(1+\sqrt{2})^2 = 3 + 2\sqrt{2}$. (0.25 pt)
			\item Factoriser le trinôme : $x^2 - (\sqrt{2} - 1)x - \sqrt{2}$. (0.75 pt)
		\end{enumerate}
	\item Étudier le signe de $\dfrac{x^2-6x +9}{3x^2 + 10x - 8}$. (2 pt)
\end{enumerate}

\underline{\large\textbf{Exercice 3 :}}(10 pt)
\begin{enumerate}
	\item Convertir en radian les mesures suivantes : $10^\circ$ ; $25^\circ$ et $300^\circ$.(0.75 pt)
	\item Soit $x\in\mathbb{R}$ tel que $\cos(x) \neq 0$.
		\begin{enumerate}
			\item Montrer que : $\sin^2(x) = \dfrac{\tan^2(x)}{1 + \tan^2(x)}$. (1 pt)
			\item En déduire les valeurs de $\sin(x)$ et $\cos(x)$ sachant que $\tan(x) = \sqrt{2}$ et $x\in\left]-\pi;-\dfrac{\pi}{2}\right[$. (2 pt)
		\end{enumerate}
	\item On considère la figure ci-contre,
		\begin{center}
			\includegraphics[scale=0.5]{Rectangle}
		\end{center}
		Donner la mesure principale de chacun des angles orientés suivants:(0.5 pt + 0.75 pt + 0.75 pt + 0.75 pt) \\
		\(
\left( \overline{\overrightarrow{AB}, \overrightarrow{BC}} \right) ; \left( \overline{\overrightarrow{AB}, \overrightarrow{CB}} \right) ; \left( \overline{\overrightarrow{AD}, \overrightarrow{AB}} \right) ; \left( \overline{\overrightarrow{BA}, \overrightarrow{DC}} \right)
\)
	\item Simplifier les expression suivantes : (0.75 pt + 0.75 pt + 1 pt + 1 pt)
	\begin{align*}
A &= \sin(\pi + x) + \cos(3\pi - x) - \sin(x - 2\pi) + \cos(x + 9\pi) \\
B &= \sin(x + 3\pi) + \sin(x + 12\pi) + \sin(x - \pi) + \sin(3\pi - x) \\
C &= \sin\left(\frac{\pi}{2} + x\right) + \sin(\pi + x) + \cos\left(\frac{3\pi}{2} - x\right) + \sin(2\pi + x) \\
D &= \cos\left(x + \frac{5\pi}{2}\right) + \cos\left(x - \frac{7\pi}{2}\right)
\end{align*}

\end{enumerate}







\end{document}